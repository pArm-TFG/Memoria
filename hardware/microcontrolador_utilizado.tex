Para el desarrollo de la placa que conforma \ac{S2} se ha empleado un microcontrolador ``dsPIC33EP512GM604''.
Los motivos por los cuales se ha usado este modelo son los siguientes: 

\begin{itemize}
 \item En primera instancia, la cantidad y la precisión de los canales \ac{PWM} que este ofrece, ya que son suficientes para poder controlar todos los motores y su precisión permite generar una señal adecuada para controlar a cada uno de ellos.

 \item Por otro lado, debido a la naturaleza de los cálculos que se deben realizar para convertir posiciones cartesianas a ángulos y viceversa, el \ac{DSP} facilita la obtención de los valores de las diferentes operaciones matriciales que permiten esta conversión.

 \item Otro aspecto importante es la posibilidad de almacenar hasta 512KB de memoria de programa.

 \item Por último, se ha elegido un \ac{DSP} de Microchip debido a que todos los integrantes del grupo de desarrollo tiene experiencia previa con este fabricante. Además, dicho fabricante proporciona documentación extensa sobre sus productos.
\end{itemize}

Una parte crítica del proyecto es la precisión y el control de los motores. En este aspecto, los canales \ac{PWM} del \ac{DSP} permiten generar ciclos de trabajo a partir de un registro de 16 bits. Para una frecuencia de $50Hz$ (período de $20ms$), un giro de 360\degree{} supondría una duración del nivel alto del \ac{PWM} de $5ms$. Es decir, se podrá controlar la rotación entre 0\degree{} y 360\degree{} con $\sfrac{2^{16}}{4} = 16384\,bits$ de control. Esto supone que obtendremos una precisión máxima de $\sfrac{360}{16384} = 0'02197\degree{}$, lo cual es suficiente para el proyecto.

Cabe destacar, que si bien los motor son capaces de girar 360\degree{} hexadecimales, debido a las limitaciones geométricas del brazo, los ángulos de giro se ven limitados en la aplicación concreta que se les dará.

En cuanto a las operaciones matemáticas, el \ac{DSP} cuenta con un circuito \ac{PLL}, el cual permite incrementar la frecuencia del oscilador para conseguir de esta manera una mayor cantidad de instrucciones por segundo y, por tanto, un mayor volumen de operaciones.

En el momento de comenzar el proyecto se podía suponer que el código no podría albergarse en la memoria que normalmente suministran microcontroladores de menores prestaciones.

Este microcontrolador, gracias a que dispone de varios puertos \ac{UART} permite recibir las ordenes y los movimientos necesarios desde \ac{S1}. Desde \ac{S2}, se realizará las operaciones matemáticas necesarias y después se encargará de generar las señales \ac{PWM} para mover los motores de tal manera que el brazo robótico quede en la posición deseada.
