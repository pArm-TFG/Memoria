En base al trabajo realizado los integrantes del equipo de desarrollo ha tomado las siguientes conclusiones técnicas 

\begin{itemize}

    \item En lo referente al \textit{framework} de desarrollo de interfaces gráficas Qt, se han podido implementar todas las funcionalidades que se han necesitado, además se han podido crear incluso funcionalidades propias, así como modificar funcionalidades ya existentes. Por tanto, se recomienda su uso para futuros proyectos.
    \item En cuanto al diseño 3D el CAD FreeCAD se presenta como una alternativa libre, gratuita y multiplataforma frente a \ac{SW} de pago. Por otro lado, destacar que su curva de aprendizaje es acusada y que en ocasiones, ocurren errores inesperados los cuales provocan, en el peor de los casos, perder trabajo no guardado y en otros caso que el trabajo ya existente se corrompa.
    \item En lo referente a la impresión 3D, la Ultimaker 3 Extended es una opción profesional y de alta calidad la cual ofrece un flujo de trabajo intuitivo y potente, permitiendo al usuario hacer configuraciones rápidas, así como editar los parámetros de impresión casi en su totalidad. Sin embargo, en piezas de reducido tamaño y con características complicadas, la impresión 3D no es la alternativa más precisa. Se recomienda, en caso de hacer un proyecto similar, trabajar con piezas de dimensiones superiores (en comparación con las utilizadas en este proyecto).
    \item Python es una alternativa libre y multiplataforma que nos ha permitido desarrollar la interfaz gráfica y la lógica del \ac{S1} con relativa sencillez y sin plantear demasiadas complicaciones. Además, la comunicación a través de la UART se realiza de forma sencilla gracias a librerías ya existente que facilitan esta labor. Se recomienda su uso para proyectos similares.
    \item MPLAB X IDE simplifica la programación de los microcontroladores de la familia Microchip ahorrando mucho tiempo de configuración, añadiendo plugins para realizar operaciones automáticamente (cálculo de valores, inicialización de puerto, etc.). Sin embargo, no es un entorno amigable con el desarrollador ya que ciertas funcionalidades como auto-completado mientras escribes, sugerencia de tipos, detección de errores mientras escribes (falta de ;, tipos de datos erróneos, funciones inexistentes, ...), etc., además de que el compilador está bloqueado en la versión gratuita, no habilitando ciertas opciones que son de pago. Al no haber alternativas no se puede usar otro entorno de desarrollo para crear código en microcontroladores Microchip.
    \item Dada la envergadura del proyecto a nivel \ac{HW}, en las etapas previas al proceso de fabricación se recomienda realizar una extensa verificación de los diagramas lógicos y físicos para evitar contratiempos en etapas posteriores, ya que estos pueden suponer soluciones sobre la marcha y, en el peor de los casos, modificar el diseño y reconstruir la placa.
    \item A pesar de la experiencia previa de parte del equipo de desarrollo con el manipulador $\mu$Arm, el fundamento matemático del proyecto tuvo que ser replanteado ya que se vio que el modelo cinemático no se ajustaba al comportamiento de la estructura física del brazo. Tras dicho estudio se obtuvo una modelización fiel del comportamiento del \pArm.
\end{itemize}