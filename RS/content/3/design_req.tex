En esta sección se describen algunas limitaciones existentes debido a distintos motivos, principalmente al \ac{HW} y estructura física del \pArm{}.

En primer lugar, existe una limitación en cuanto a los materiales de fabricación de la estructura física del brazo, ya que se quiere construir mediante la combinación de dos materiales plásticos: \ac{PLA} y \ac{CPE}. El primer material se utiliza para impresión en 3D y, dado que el \pArm{} se quiere imprimir por piezas utilizando una impresora de este tipo, el \ac{PLA} es un material adecuado.
El \ac{CPE} por su parte ofrece una alta resistencia a productos químicos y, lo que es
más importante para este brazo robótico, una gran resistencia a temperaturas elevadas
y a la fricción, lo que lo hace en un material ideal para diseñar e imprimir estructuras
mecánicas \cite{ultimakerCPEFamilyUltimaker}.

Por otro lado, para simplificar los cálculos en el modelo dinámico, se ha optado por usar un manipulador robótico pantográfico. Este tipo de manipuladores tienen una estructura similar a un flexo y la principal ventaja es que los motores se encuentran muy cercanos a la base. De esta forma, el peso de los mismos no debe ser desplazado al realizar movimientos en las articulaciones del brazo.
