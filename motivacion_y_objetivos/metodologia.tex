Dado que se pretende hacer un desarrollo de ingeniería completo, la metodología es un
punto muy importante en este proyecto.

Primeramente, antes de hacer ningún tipo de desarrollo o implementación, se hizo un
estudio del problema, y de lo que se pretendía obtener. Por una parte, se comprobó
hasta qué punto podrían ser reutilizables los diseños provistos por UFACTORY
en su página de GitHub. Esto permitió diseñar elementos nuevos, adaptar los recursos
a lo que hay disponible, etc. 
Por otro lado, se estudió qué placa de control se quería utilizar. Debido a la familiaridad
de los integrantes del equipo con las placas de la familia ``Microchip'', se plantearon
distintas alternativas:

\begin{itemize}
    \item Controladores de gama media de la familia PIC16F.
    \item Controladores de gama superior de la familia PIC32F.
    \item Controladores digitales de la señal, de la familia dsPIC.
\end{itemize}

Se optó por utilizar los últimos mencionados, ya que disponen de un control específico
matricial y matemático para poder agilizar las operaciones realizadas, de forma que
los cómputos necesarios se podrían realizar íntegramente en el microcontrolador.

Además, se estudió cómo se quería plantear la comunicación con el brazo: de forma
completamente autónoma o mediante un equipo auxiliar. Para evitar la complejidad extra
que habría surgido de desarrollar un sistema de control completamente autónomo del brazo
por sí solo, se decidió conectarlo a un equipo auxiliar externo que lo gobierne, y que
el \pArm{} no funcionase si no es estando conectado.

Una vez se definieron estos puntos, se pasó al diseño lógico del sistema que deberán
tener tanto \ac{S1} como \ac{S2}, mediante especificación de requisitos, diagramas lógicos,
diagramas físicos, diagramas de diseño, etc. Esta parte del proyecto es de las más 
importantes, ya que sustenta las ideas y las funcionalidades que habrán de estar presentes
en el producto final. Mientras tanto, se han ido desarrollando pruebas y mecanismos de
control para ir asegurando la correcta calidad del trabajo.

Finalmente, una vez completada esta parte de diseño, se pasa a la implementación real.
Dada la situación del COVID--19, esta fase de implementación se ha retrasado sobremanera,
impidiendo pues presentar el proyecto en el mes de julio, como estaba previsto, y teniendo
que acotar los plazos de implementación a, posiblemente, un mes. En el momento de
implementación, se creará la placa diseñada y se empezará la impresión de distintas piezas
3D, para comprobar su funcionamiento en conjunto e ir solucionando los posibles errores
que aparezcan.

Durante este proceso, se ha ido desarrollando además de forma paralela la memoria que 
acompaña el proyecto, permitiendo ir actualizándola con los últimos cambios y mejoras que
se han considerado de interés para aparecer descritas.