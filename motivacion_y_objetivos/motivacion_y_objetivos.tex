Durante el primer semestre del cuarto curso de Ingeniería de Computadores,
hay dos asignaturas las cuales propiciaron el desarrollo de este proyecto: robótica
y sistemas empotrados.

Con la primera, se vio la potencia de los brazos robóticos y se desarrolló un estudio
sobre un controlador el cual se ha hablado con anterioridad: el $\mu$Arm \cite{UPMRoboticsUarm2019b}.
Con la segunda asignatura, se vio cómo con sistemas de aplicación específica se podían
desarrollar circuitos con suficiente potencia como para poder tomar el control de otros
dispositivos más grandes y complejos aplicando la lógica estudiada a lo largo de los
años.

Se tomaron en cuenta los conocimientos obtenidos de las asignaturas anteriores
para empezar un desarrollo que uniera esos dos campos: diseñar un brazo robótico
impreso en 3D el cual estuviera gobernado por un microcontrolador en una placa de
control de propósito específico. Para ello, se parte de los diseños 3D provistos
en la web de UFACTORY \cite{UFACTORYXArmTextbackslashtextbaruArm} para su posterior
adaptación y reutilización. En lo referente a la placa de control, el brazo original
utiliza una placa Arduino Mega \cite{ArduinoMega2560}, por lo que se decidió (para dar
mayor peso a la parte de ingeniería e intentar reducir costes) diseñar e implementar al completo una placa con
otro microcontrolador para gobernar dicho brazo robótico.

Principalmente, este trabajo se desarrolla bajo las dos perspectivas siguientes:
\begin{itemize}
    \item Aplicar en un proyecto de ingeniería real las competencias y técnicas que 
    se han ido aprendiendo a lo largo de los distintos cursos del Grado de Ingeniería
    de Computadores (61CI).
    \item Construir una alternativa asequible y accesible, tanto a niveles de \ac{OS} y
    \ac{OH}, de un brazo robótico de manera que cualquier persona interesada en este
    ámbito de la ingeniería pueda introducirse y aprender, e incluso montar el brazo
    por sí mismo.
\end{itemize}

Para la primera perspectiva, la forma de afrontarla y desarrollarla está detallada en el
punto siguiente (\ref{sec:methodology}). Para la perspectiva de desarrollo de un
producto accesible y asequible, se partió desde el abaratamiento de costes: el brazo
original $\mu$Arm se encuentra disponible en venta por aproximadamente \$749. Dicho precio,
pese a no ser especialmente elevado, impide a muchas personas el acceso a la robótica
en un brazo que pretende ser educativo y útil. Por este motivo, se desarrolla este proyecto
principalmente para resultar barato. Además, siguiendo con la política del brazo
original, el proyecto se desarrolla bajo las premisas \ac{OS} y \ac{OH}, de manera que
inclusive para aquellos que no puedan imprimir el brazo 3D se dispone de forma universal
todos los diagramas, planos, esquemas, diseños y código fuente que se ha empleado para
acabar desarrollando el brazo \pArm{}.