\chapter{Código fuente ``\pArm{} \textit{configurator}''}
\label{anex:pArm-configurator}
\lstinputlisting[label={lst:dh_table_py}, style=Python]{pArm-configurator/src/manipulator/dh_table.py}
\lstinputlisting[label={lst:manipulator_py}, style=Python]{pArm-configurator/src/manipulator/manipulator.py}
\section{Enlace a \textit{Jupyter Notebook} para configurar el \pArm}
\label{anex:jupyter_binder}
\url{https://s.javinator9889.com/pArm-config}\qquad
\qrcode{https://s.javinator9889.com/pArm-config}

\chapter{Código fuente S2}
\label{anex:source_code}
\section{\textit{Header files}}
\lstinputlisting[label={lst:init_h}, style=C]{pArm-S2/pArm.X/init.h}
\lstinputlisting[label={lst:interrupts_h}, style=C]{pArm-S2/pArm.X/interrupts.h}
\lstinputlisting[label={lst:pragmas_h}, style=C]{pArm-S2/pArm.X/pragmas.h}
\lstinputlisting[label={lst:system_types_h}, style=C]{pArm-S2/pArm.X/system_types.h}
\lstinputlisting[label={lst:defs_h}, style=C]{pArm-S2/pArm.X/utils/defs.h}
\lstinputlisting[label={lst:time_h}, style=C]{pArm-S2/pArm.X/utils/time.h}
\lstinputlisting[label={lst:uart_h}, style=C]{pArm-S2/pArm.X/utils/uart.h}
\lstinputlisting[label={lst:utils_h}, style=C]{pArm-S2/pArm.X/utils/utils.h}
\lstinputlisting[label={lst:types_h}, style=C]{pArm-S2/pArm.X/utils/types.h}
\lstinputlisting[label={lst:servo_h}, style=C]{pArm-S2/pArm.X/motor/servo.h}
\lstinputlisting[label={lst:motor_h}, style=C]{pArm-S2/pArm.X/motor/motor.h}

\section{\textit{Source files}}
\lstinputlisting[label={lst:init_c}, style=C]{pArm-S2/pArm.X/init.c}
\lstinputlisting[label={lst:interrupts_c}, style=C]{pArm-S2/pArm.X/interrupts.c}
\lstinputlisting[label={lst:time_c}, style=C]{pArm-S2/pArm.X/utils/time.c}
\lstinputlisting[label={lst:uart_c}, style=C]{pArm-S2/pArm.X/utils/uart.c}
\lstinputlisting[label={lst:servo_c}, style=C]{pArm-S2/pArm.X/motor/servo.c}
\lstinputlisting[label={lst:motor_c}, style=C]{pArm-S2/pArm.X/motor/motor.c}

\chapter{Especificación de requisitos}
\label{anex:requirements}
\subsection{Introducción}
\label{ch:intro}
\input{RS/content/1/introduction.tex}

\subsubsection{Propósito}
\input{RS/content/1/purpose.tex}

\subsubsection{Alcance}
\input{RS/content/1/scope.tex}

% \subsection{Definiciones, siglas, y abreviaturas}
% \input{RS/content/1/defs.tex}

\subsubsection{Visión global}
\input{RS/content/1/global_scope.tex}

\subsection{Descripción general}
\label{Descripción general}

\subsection{Perspectiva del producto}
\input{RS/content/2/product_perspective.tex}

\subsection{Funciones del producto}
\input{RS/content/2/product_functions.tex}

\subsection{Características del usuario}
\input{RS/content/2/user_specs.tex}

\subsection{Restricciones}
\input{RS/content/2/restrictions.tex}

\subsection{Supuestos y dependencias}
\input{RS/content/2/dependencies.tex}

\subsection{Requisitos pospuestos}
\input{RS/content/2/posposed_requirements.tex}

\newpage

\subsection{Requisitos específicos}
\label{Requisitos específicos}

%\subsection{User Interfaces}

\subsection{Requisitos de la interfaz externa}
\input{RS/content/3/user_interface.tex}
\input{RS/content/3/hardware_interface.tex}
\input{RS/content/3/communications_interface.tex}

\subsection{Casos de uso}
\input{RS/content/3/user_cases.tex}

\subsection{Requisitos funcionales}
\input{RS/content/3/requirements}
% \subsubsection{Requisitos \textit{software}}
% \input{content/3/software_requirements.tex}
% \input{content/3/hardware_requirements.tex}

% \subsubsection{Requisitos \textit{hardware}}
% \input{content/3/hardware_requirements.tex}

% \subsubsection{Requisitos de rendimiento}
% \input{content/3/performance_req.tex}

\subsection{Restricciones del diseño}
\input{RS/content/3/design_req.tex}

\subsection{Atributos del sistema \textit{software} y \textit{hardware}}
\input{RS/content/3/system_attrs.tex}

\subsection{Requisitos no funcionales}
\input{RS/content/3/non_functional_req.tex}

% \appendix
% \input{RS/content/appendix/appendix.tex}


\newpage
\chapter{Matriz pseudo--inversa cuando $\left|J\left(\dot{q}\right)\right| = 0$}
\label{anex:pinv}
La matriz pseudo--inversa se puede ver desde el siguiente enlace:
\url{https://raw.githubusercontent.com/pArm-TFG/Memoria/master/pictures/equation.svg}\qquad
\qrcode{https://raw.githubusercontent.com/pArm-TFG/Memoria/master/pictures/equation.svg}

\chapter{Código fuente \ac{S1}}
\lstinputlisting[label={lst:connection_py}, style=Python]{pArm-S1/pArm/communications/connection.py}

\lstinputlisting[label={lst:control_py}, style=Python]{pArm-S1/pArm/control/control.py}
\lstinputlisting[label={lst:control_interface_py}, style=Python]{pArm-S1/pArm/control/control_interface.py}
\lstinputlisting[label={lst:control_management.py}, style=Python]{pArm-S1/pArm/control/control_management.py}
\lstinputlisting[label={lst:heart_beat.py}, style=Python]{pArm-S1/pArm/control/heart_beat.py}

\lstinputlisting[label={lst:generator_py}, style=Python]{pArm-S1/pArm/gcode/generator.py}
\lstinputlisting[label={lst:interpreter_py}, style=Python]{pArm-S1/pArm/gcode/interpreter.py}

%ALEJANDRO, METE AQUI TU CODIGO

\lstinputlisting[label={lst:logger_py}, style=Python]{pArm-S1/pArm/logger/logger.py}

\lstinputlisting[label={lst:logger_py}, style=Python]{pArm-S1/pArm/security/rsa.py}

\lstinputlisting[label={lst:error_data_py}, style=Python]{pArm-S1/pArm/utils/error_data.py}