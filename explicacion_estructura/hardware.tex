Los elementos \textit{hardware} conforman la implementación física del manipulador y de todos los componentes empleados para controlarlo.

En términos generales, el \textit{hardware} usado en el proyecto se descompone en diferentes elementos:
\begin{itemize}
    \item Impresión 3D de la estructura física del manipulador.
    \item Motores empleados en el manipulador.
    \item Desarrollo de la placa de circuito impreso de control y microcontrolador empleado.
    \item Comunicaciones entre los diferentes subsistemas.
\end{itemize}

En primer lugar, la impresión 3D es la tecnología seleccionada para la fabricación de la estructura física del manipulador debido a su bajo coste y accesibilidad. Esta parte del proyecto se centra en llevar a cabo la fabricación y construcción de la estructura física del manipulador, así como su ensamblado y testeo. Este apartado se ubica en el apartado 5.1 de la memoria.

En segundo lugar, la elección de los motores que dotan de movilidad a la estructura es una decisión crucial y que depende principalmente de cuales sean las características físicas del manipulador, así como de las tareas que se quieran realizar con el mismo. Existen numerosas opciones en cuanto a motores, por ejemplo, motores DC, servomotores, motores paso a paso, etc. Este apartado se ubica en el apartado 5.5 de la memoria.

En tercer lugar, el desarrollo de la PCB de control y elección del microcontrolador representan la parte más importante dentro del bloque hardware del proyecto. El objetivo principal de esta parte del proyecto es llevar a cabo el diseño y construcción de una PCB personalizada, adaptada especialmente a los actuadores y microcontrolador usados para llevar a cabo el control del movimiento del manipulador. Se considera que esta PCB representa uno de los elementos hardware esenciales para el correcto desarrollo del proyecto. Este apartado se ubica en el apartado 5.3 de la memoria.

En último lugar, el diseño e implementación de los canales de comunicación y protocolos necesarios para comunicar los dos subsistemas principales requiere desarrollo hardware y software de forma equitativa, además, también representa uno de los elementos cruciales del proyecto. Este apartado se ubica en el apartado 5.4 de la memoria.




