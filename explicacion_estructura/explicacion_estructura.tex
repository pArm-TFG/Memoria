En este apartado se detalla cual es la estructura del proyecto y la memoria, así como los principales bloques que componen el desarrollo y construcción del brazo robótico.

El diseño y construcción de un brazo robótico es un proceso multidisciplinar en el que se deben emplear diversas áreas del conocimiento. Desde un primer momento, este proyecto se postuló como un proyecto completo de ingeniería, y es precisamente por eso que el proyecto está dividido en varios bloques, los cuales desempeñan una función clave en el desarrollo correcto del mismo.

El proyecto está divido en tres grandes bloques: modelo matemático, elementos hardware y elementos software. Cada una de estas partes se encuentra a su vez subdivida en diferentes partes o hitos, sin embargo, no es necesario describirlos con tanta precisión por el momento para poder comprender la estructura completa del proyecto.

Cabe destacar que, desde un punto de vista de ingeniería, a cada uno de los grandes bloques anteriormente mencionados se le puede asociar una función dentro del proyecto:
\begin{itemize}
    \item El modelo matemático es la parte mas teórica del proyecto y su función es la de aportar una base formal y lógica que permita realizar cálculos sobre la cinemática del brazo robot. Este bloque se encuentra ubicado en el apartado 4 de la memoria.
    
    \item Los elementos hardware del proyecto constituyen la realidad física del brazo robótico y están estrictamente  relacionados con la construcción del mismo, así como con el control de los actuadores y demás elementos hardware a nivel físico. Este bloque se encuentra ubicado en el apartado 5 de la memoria.
    
    \item Los elementos software del proyecto constituyen el principal mecanismo para implementar el modelo matemático y la lógica de funcionamiento del sistema completo, a través de la programación de los elementos hardware. Este bloque se encuentra ubicado en el apartado 6 de la memoria.
\end{itemize}

En cada uno de los bloques anteriores, ya sean hardware o software y requieran construcción física o implementación mediante programación, se incluye la  realización de pruebas de funcionamiento así como las revisiones pertinentes.

Es importante remarcar que, debido a la complejidad del sistema, el mismo esta divido en dos subsistemas que aglutinan funcionalidades vitales para el correcto funcionamiento del manipulador robótico:
\begin{itemize}
    \item El sistema uno o S1 esta formado por la aplicación gráfica de usuario que se utiliza para controlar manualmente los movimientos del robot. Este subsistema es un elemento software íntegramente y está descrito en el apartado 6.1 de la memoria.
    
    \item El sistema dos o S2 esta formado por la estructura física del manipulador, los actuadores y la placa de circuito impreso de control. Este subsistema combina elementos hardware y software, así como conceptos del modelo matemático. Los elementos software se describen en el apartado 6.2 de la memoria, mientras que los elementos hardware se describen en el apartado 5 de la memoria.
\end{itemize}

A continuación, se describen de forma detallada todos los bloques descritos anteriormente y a su vez, se mencionan las principales subdivisiones de cada uno de ellos.\\
