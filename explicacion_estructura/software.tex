Los elementos \textit{software} del proyecto abordan los siguientes aspectos:

\begin{itemize}
    \item Desarrollo de la aplicación de control del brazo robótico, implementada mediante una interfaz gráfica de usuario para garantizar su accesibilidad y facilidad de uso. Esta implementación se lleva a cabo en S1.
    \item Programación del microcontrolador e implementación del modelo matemático en la práctica con el objetivo de controlar los movimientos del brazo robótico. Esta implementación se lleva a cabo en S2.
\end{itemize}

En primer lugar, mediante el desarrollo de la aplicación de usuario se busca ofrecer una forma de controlar los movimientos del robot de forma fácil y accesible, para ello se ha desarrollado una interfaz de usuario que se ejecuta en un ordenador auxiliar. Desde esta aplicación el usuario puede controlar los movimientos del robot, además de monitorizar el estado del mismo. Las órdenes dadas por el usuario son enviadas al microcontrolador para su ejecución mediante los canales de comunicación mencionados anteriormente. Este desarrollo se ha llevado a cabo mediante el lenguaje de programación Python. Este apartado se ubica en el apartado 6.1 de la memoria.

En segundo lugar, la programación del microcontrolador representa una parte esencial del proyecto, ya que toda la lógica de funcionamiento y control de los actuadores del brazo robótico se lleva a cabo en el mismo. Es por ello que la labor principal del microcontrolador es orquestar el funcionamiento de los actuadores, así como de realizar el computo necesario para transformar las ordenes del usuario en movimientos consecuentes del brazo robótico. La programación del microcontrolador se ha llevado a cabo mediante el lenguaje C. Este apartado se ubica en el apartado 6.2 de la memoria.
