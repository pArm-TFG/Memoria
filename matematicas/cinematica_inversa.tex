La cinemática inversa se presenta como lo ``más cercano'' a nuestro mundo y a nuestra
forma de actuar. El ser humano, como ser tridimensional, se mueve mediante coordenadas
cartesianas conformadas por puntos definidos en espacio definido por los planos de los
ejes $XYZ$, pero no se desenvuelve con la misma soltura con las coordenadas angulares.
Cuando se realiza un giro en alguno de los brazos no se hace pensando: ``\textit{voy a mover el codo
    15\textdegree{} a la derecha y el hombro 23\textdegree{} a la izquierda y así coloco la
    mano justo donde quiero}'' sino que directamente se visualiza el movimiento que se pretende
hacer, a dónde se quiere mover el brazo y se articulan los músculos para
colocarlo en esa posición.

Por esto, cuando manipulamos un brazo robótico resulta más sencillo indicar a dónde
se quiere que vaya el \textit{end--effector} del brazo más que cúanto ha de rotar cada uno de los
motores. De esta forma, el estudio de la cinemática inversa se convierte en una de las
partes más importantes del modelo matemático de cualquier manipulador.

El problema surge en tanto que la cinemática inversa, a diferencia de la cinemática
directa, no dispone de ecuaciones que permitan obtenerla directamente. Si bien en el
punto anterior se vio cómo, a partir de una tabla de \textit{Denavit--Hartenberg},
se calculaban las matrices que permiten obtener tanto las traslaciones como las rotaciones
relativas y, al multiplicarlas, la traslación absoluta $T$ y la rotación absoluta $R$,
en la cinemática inversa no hay ningún modelo matemático que permita una aproximación
directa genérica para cualquier manipulador.

A raíz de lo anterior, se plantean así dos maneras para poder obtener la relación entre
coordenadas cartesianas y coordenadas articulares:

\begin{enumerate}
    \item Mediante fuerza bruta. Como la obtención de la cinemática directa es siempre
          igual, según la precisión que se busque obtener a nivel de coordenadas cartesianas
          se puede plantear la opción de realizar un mapa de puntos: para un conjunto de
          coordenadas articulares $\left\{\theta_0^i,\theta_1^j,\cdots,\theta_n^k\right\}$ se
          obtienen unas coordenadas cartesianas
          $\left\{x^{ij\cdots k}, y^{ij\cdots k}, z^{ij\cdots k}\right\}$
          (donde $i,j,\cdots,k$ representan unos ángulos en específico).

          De esta manera, para el \pArm{} en específico, se tienen
          $\left\{\theta_1, \theta_2, \theta_3\right\}$ y en total, suponiendo una precisión
          de un decimal considerando además un espectro de giro de $\left[0,180\right]\degree$,
          se disponen de una combinación de $1800^3$ posibles ángulos, lo que se traduce
          en un mapa de $\numprint{5832000000}$ ángulos que generan la misma cantidad de posiciones
          en $XYZ$. Si se quisieran usar dos decimales de precisión en ángulos (ya que hay motores capaces
          de ello), se tendrían pues $\numprint{5.832e12}$ combinaciones de ángulos y puntos.

    \item Mediante el cálculo numérico y el razonamiento matemático. Como no hay una
          ecuación genérica que permita el cálculo de la inversa, cualquier cálculo numérico ha
          de ser previamente razonado y estudiado. La aproximación a la cinemática inversa
          mediante este método es costosa y pueden haber situaciones en las que no resulte
          viable debido a la inversión en tiempo y coste: estudiar las distintas posiciones
          a las que puede llegar el manipulador, estudio de los puntos críticos del mismo
          así como plantear, si es necesario, soluciones para puntos con múltiples soluciones
          (aquellos a los que se puede llegar con combinaciones de los ángulos de entrada
          distintas).
\end{enumerate}

A la hora de desarrollar la inversa, se ha de escoger entre alguna de las dos aproximaciones
anteriores, teniendo en cuenta principalmente distintos criterios que pueden marcar la
diferencia entre uno y otro:

\begin{itemize}
    \item Por una parte, el rendimiento: el modelo matemático suele ser en general
          bastante eficiente en lo que a tiempo de cálculo se refiere, pero siempre va
          supeditado al manipulador que representa. Esto es, manipuladores con más
          grados de libertad implican en general un modelo matemático mucho más complejo que según
          las operaciones que implique puede no ser viable para el sistema en que se va a ejecutar.

          Por otro lado, un mapa por su estructura y organización siempre permite el acceso a las
          claves y sus valores bajo un $\bigO{1}$, haciéndolos la mejor opción en términos
          de eficiencia si se busca una ejecución rápida.

    \item Por otra, la memoria: un mapa siempre requiere de mucha más memoria que
          una primitiva u otra estructura de datos. Principalmente se debe a su organización
          en memoria ya que, además de las claves y sus valores, se debe guardar un \textit{hash}
          o un \textit{set} (según esté implementada la librería) de todas y cada una de las
          claves para garantizar así que el tiempo de acceso sea $\bigO{1}$.
          Además, el mapa tendría que ir guardado o directamente en el espacio de código
          (y copiado a la \ac{RAM} en tiempo de ejecución) o bien guardado en un fichero
          binario para su posterior carga en el sistema durante la ejecución, lo cual implica
          que sería necesario contar con ese espacio en el sistema de ficheros donde se guarde.

          En cambio, el modelo matemático carece de este problema ya que se utilizan
          principalmente primitivas y operaciones matemáticas que se realizan directamente
          sobre un co-procesador, si existe, o sobre el procesador en sí. Aunque se puedan
          usar muchas primitivas, es difícil que alcancen en tamaño en memoria a un mapa.

    \item Además, hay que tener en cuenta el esfuerzo de la obtención. La aproximación
          por fuerza bruta requiere de bastante tiempo para la obtención del mapa al completo.
          Además, un cambio en la cantidad de decimales implicaría un recálculo casi completo del mapa
          con un aumento de tiempo exponencial, aunque se puede automatizar y que sea realizado
          por otro equipo.

          Sin embargo, dado que el modelo matemático requiere de un razonamiento y estudio
          de tanto las características geométricas del manipulador como de las interacciones
          entre los elementos del mismo, el tiempo es en principio desconocido. Depende
          directamente de las aptitudes tanto matemáticas como técnicas del equipo trabajando
          en ello y, además, la verificación, comprobación y validación de los resultados
          obtenidos puede implicar tener que replantearlo y modelarlo de nuevo, necesitando
          así de más tiempo hasta que se consigan resultados conformes a los requisitos
          establecidos.
\end{itemize}

Para este proyecto se ha preferido hacer el modelo matemático ya que se plantearon las
características del modelo por fuerza bruta pero fue descartado debido a una estimación de
uso de memoria excesivo (no habría sido suficiente según la disponible en el dispositivo\footnote{teniendo en cuenta
    que habría sido necesario guardar tuplas de tres elementos por clave junto con tuplas de
    otros tres elementos para el valor, donde cada elemento sería de tipo \texttt{float}
    (lo que se traduce en $\numprint[B]{4}$ por elemento), habría supuesto un uso de
    aproximadamente: $\left(\numprint{5.382e9}\right)^2 \cdot \numprint[B]{4} = \numprint[B]{1.158e20} \approx \numprint[TB]{1.158e11} $,
    (suponiendo que las tuplas no usan espacio adicional) lo cual es inviable para el sistema.}).
Además, dado que se cuenta con un procesador con gran capacidad de cómputo, las operaciones
matemáticas se realizan a una gran velocidad y en particular las multiplicaciones, ya que
se cuenta con un conjunto de instrucciones y con una \ac{ALU} que permiten su realización
a la misma velocidad que una suma con números de hasta 16 bits \cite{microchipDsPIC33EPIC24EFRM2010}.

Para plantear la cinemática inversa del \pArm{} se han de distinguir dos partes:
\begin{itemize}
    \item La base $\left(\theta_0\right)$, que rota sobre el eje $Y$ y cuyo movimiento no está supeditado al
          del resto de motores.

    \item El triángulo superior, conformado por $\left\{\theta_1, \theta_2\right\}$ donde
          los ambos ángulos dependen de la posición final y están directamente relacionados.
\end{itemize}

Por otra parte, dada la configuración geométrica del robot, sabemos las siguientes premisas:

\begin{itemize}
    \item $x$ se encuentra comprendido en el rango $\left(0, A_{M_L}\right]$, donde $A_{ML}$
          es ``\textit{Arm Maximum Length}'' y viene definido por la ecuación \ref{eq:aml}:

          \begin{equation}\label{eq:aml}
              A_{M_L} = \left(\overline{A_L} + \overline{A_U}\right) \cdot \cos{\theta^{LU}_{Max}} + A_{EF_L}\\
          \end{equation}

          donde cada uno de los elementos anteriores representan:

          \begin{equation*}
              \left\{\begin{aligned}
                  \overline{A_L}    & \equiv \text{``\textit{Arm Lower}''} = \numprint[mm]{142}                 \\
                  \overline{A_U}    & \equiv \text{``\textit{Arm Upper}''} = \numprint[mm]{158.8}               \\
                  \theta^{LU}_{Max} & \equiv \widehat{A_L A_U}_{Max} = \frac{13\pi}{15}~rad                     \\
                  A_{EF_L}          & \equiv \text{``\textit{Arm End--Effector Length}''} = \numprint[mm]{44.5} \\
              \end{aligned}
              \right.
          \end{equation*}

    \item $y$ por su parte se encuentra comprendido en el rango $\left[-A_{M_L}, A_{M_L}\right]$,
          donde $A_{M_L}$ está definido en la ecuación anterior (ecuación \ref{eq:aml}).
    \item $z$ pertenece al rango $\left[0, A_{M_H}\right]$, donde $A_{M_H}$ es
          ``\textit{Arm Maximum Height}'' y viene definido por la ecuación \ref{eq:amh}:

          \begin{equation}\label{eq:amh}
              A_{M_H} = A_{B_H} + \overline{A_L} + \overline{A_U} \cdot \sin{^{Max}\theta^{A_U}_{A_L \parallel A_B}}
          \end{equation}

          donde los elementos anteriores representan:
          \begin{equation*}
              \left\{\begin{aligned}
                  A_{B_H}                                & \equiv \text{``\textit{Arm Base Height}''} = \numprint[mm]{106.1}                       \\
                  ^{Max}\theta^{A_U}_{A_L \parallel A_B} & \equiv \text{``Ángulo máximo de $A_U$ cuando $A_L \parallel A_B$''} = \frac{\pi}{8}~rad \\
              \end{aligned}
              \right.
          \end{equation*}
\end{itemize}

De esta forma, tenemos que:

\begin{align*}
    x & \in \left(0, A_{M_L}\right]        \\
    y & \in \left[-A_{M_L}, A_{M_L}\right] \\
    z & \in \left[0, A_{M_H}\right]        \\
\end{align*}

Una vez definidas las características anteriores, se puede empezar a obtener los distintos
ángulos. Por una parte, la obtención de $\theta_0$ se puede realizar directamente como
se muestra en la ecuación \ref{eq:t0}:

\begin{equation}\label{eq:t0}
    \theta_0 = \left\{\begin{aligned}
        \pi - \arctan{\frac{x}{y}} & ,~y > 0 \\
        - \arctan{\frac{x}{y}}     & ,~y < 0 \\
        \frac{\pi}{2}              & ,~y = 0
    \end{aligned}
    \right.
\end{equation}

La obtención de los dos ángulos restantes $\left\{\theta_1, \theta_2\right\}$ es más
compleja y se han de realizar previamente ciertas modificaciones en los puntos de entrada.

La idea principal radica en plantear la estructura superior del brazo como un triángulo
y, mediante operaciones y leyes trigonométricas, ir obteniendo distintos parámetros hasta
finalmente conseguir los ángulos finales. A modo de guía, se muestra en la figura
\ref{fig:ik_over_arm} una representación de cómo está el triángulo sobre el brazo
robótico.

\begin{figure}[H]
    \centering
    \includegraphics[width=.5\linewidth]{pictures/ik_cosine_law_over_arm.png}
    \caption{Parte superior del \pArm{} con el triángulo para obtener la cinemática inversa.}
    \label{fig:ik_over_arm}
\end{figure}

La distribución en particular del triángulo permite aplicar el teorema del coseno (ecuación
\ref{eq:cosine_law}) y obtener así los lados o bien los ángulos.

\begin{equation}\label{eq:cosine_law}
    c^2 = a^2 + b^2 - 2ab\cos{\gamma}
\end{equation}

En particular, para el brazo robótico se conocen siempre el tamaño de los lados
(según la figura \ref{fig:ik_over_arm}) `$a$' y `$c$', pero el lado `$b$' es variable. Según las
medidas del brazo se tiene que:

\begin{equation*}
    \left\{
    \begin{aligned}
        a & = \numprint[mm]{142}   \\
        c & = \numprint[mm]{158.8}
    \end{aligned}
    \right.
\end{equation*}

Como dicho lado `$b$' es desconocido en principio y además los ángulos $\left\{\gamma, \beta, \alpha\right\}$
también lo son, no se podría aplicar el teorema del coseno ya que se necesita cumplir alguna de las
siguientes condiciones:

\begin{itemize}
    \item Se conocen dos lados y el ángulo entre ellos.
    \item Se conocen los tres lados.
    \item Se conocen dos lados y el ángulo opuesto a ellos.
\end{itemize}

Dado que conocemos dos de los lados siempre y, según las características de un triángulo,
los ángulos no varían si y solo si la proporción de los lados es la misma tras aplicar
distintas operaciones que modifiquen su tamaño, se puede redimensionar el triángulo
mostrado en la figura \ref{fig:ik_over_arm} con la intención de buscar que el lado
desconocido sea unitario, esto es, mida $\numprint[mm]{1}$, pudiendo aplicar así en particular una forma del
teorema del coseno que permite obtener el ángulo `$\gamma$' teniendo los tres lados $\left\{a,b,c\right\}$
(ecuación \ref{eq:cosine_law_angle}):

\begin{equation}\label{eq:cosine_law_angle}
    \gamma = \arccos{\left(\frac{a^2 + b^2 - c^2}{2ab}\right)}
\end{equation}

\begin{figure}[H]
    \centering
    \includegraphics[width=.3\linewidth]{pictures/cosine_law.png}
    \caption{Triángulo para la aplicación del teorema del coseno en la ecuación \ref{eq:cosine_law_angle}.}
    \label{fig:cosine_law_triangle}
\end{figure}

Sabiendo que los puntos que varían y que afectan a los ángulos
$\theta_1, \theta_2$ son `$x$' y `$z$' se ha de buscar que entonces uno de los segmentos
del triángulo sea de la forma $\sqrt{x^2 + z^2}$ (el módulo del vector $\left|{\overrightarrow{xz}}\right|$).
Esto permitiría que el triángulo cambiase de forma según la extensión del brazo.
Por otra parte, como se conocen las longitudes que conforman los segmentos del brazo, se
ha de buscar que además uno de los lados tenga un tamaño que se base en dichos segmentos,
dejándolo también fijo. Dicho tamaño no ha de ser arbitrario sino que, para ser coherente
con la estructura superior del brazo, debe implicar a los elementos que lo conforman
$\left(a_2, a_3\right)$.
En este caso, el otro lado fijado se busca que esté definido por la proporción:
$\nicefrac{\overline{A_U}}{\overline{A_L}}$.
Finalmente, con esos dos largos ya predefinidos, las variables `$x$' y `$z$' han de ser
modificadas con las variaciones que se han aplicado. Como en principio
uno de los lados se busca que sea unitario y el otro se ha definido por $\nicefrac{\overline{A_U}}{\overline{A_L}}$,
se tendrán que multiplicar dichas variables por `$1$' y dividirlas por $\overline{A_L}$
para conservar la proporción, además de realizar distintas operaciones para eliminar
la altura de la base, las traslaciones en los ejes y la altura del \textit{end--effector}.

Para empezar, se modifica el valor de entrada de $z$ obteniéndose así $z'$ en donde
primeramente hay que añadir el valor de $h_o$ (\textit{height offset}). Dicho \textit{offset}
varía según el \textit{end--effector} que esté en uso, habiendo disponibles \cite{UArmDeveloperSwiftProForArduino}:

\begin{itemize}
    \item $\numprint[mm]{74.55}$ para el \textit{end--effector} normal.
    \item $\numprint[mm]{51.04}$ para el cabezal láser.
    \item $\numprint[mm]{74.43}$ para el cabezal 3D.
    \item $\numprint[mm]{74.43}$ para el cabezal con bolígrafo.
    \item $\numprint[mm]{16.01}$ si no hay ningún \textit{end--effector} conectado.
\end{itemize}

Para el \pArm{}, en esta primera versión del desarrollo, se establece $h_o = \numprint[mm]{16.01}$.

\begin{equation}
    z' = z + h_o
\end{equation}

Además, para tener un triángulo con los lados y ángulos necesarios, se ha de quitar la
altura que añade la base a $Z$ y se divide el valor por la longitud del brazo inferior
$\overline{A_L}$ (ecuación \ref{eq:z_mod}):

\begin{equation}\label{eq:z_mod}
    z' = \frac{z' - A_{B_h}}{\overline{A_L}},
    \left\{\begin{aligned}
        A_{B_h}        & \equiv \text{``\textit{Arm Base Height}''} = \numprint[mm]{106.1} \\
        \overline{A_L} & \equiv \text{``\textit{Arm Lower}''} = \numprint[mm]{142}
    \end{aligned}\right.
\end{equation}

Para $x$ se ha de ``desproyectar'' el valor sobre el eje (obteniendo de esta manera el
valor real del vector $\overrightarrow{x}$ para poder usarlo en el cálculo de
$\left|\overrightarrow{xz}\right|$) y eliminar las desviaciones
que se han añadido tanto por el \textit{end--effector} como por la base en sí (en la
figura \ref{fig:sizes} se observa una desviación en la base la cual se contempla
en la tabla \ref{tab:uArm-ld-values}). Estas modificaciones se muestran en la ecuación
\ref{eq:x_mod}:

\begin{equation}\label{eq:x_mod}
    x' = \frac{\frac{x}{\sin{\theta_0}} - T_X - \overline{A_{B_D}}}{\overline{A_L}}
\end{equation}

Además, al igual que se hizo en la ecuación \ref{eq:z_mod} y para mantener la proporción
entre los lados, se divide el valor por $\overline{A_L}$.

Con las modificaciones anteriores se ha conseguido obtener lo que se propuso
inicialmente:

\begin{itemize}
    \item Existe un lado variable definido por: $\sqrt{x'^2 + z'^2}$ que permite modificar
          el triángulo para obtener los valores reales de los ángulos $\left\{\theta_1, \theta_2\right\}$.
    \item Hay un lado fijo que está definido por la proporción de los dos segmentos
          que conforman la parte superior del brazo y definido por: $\nicefrac{\overline{A_U}}{\overline{A_L}}$.
    \item El lado desconocido tiene un tamaño fijo de $\numprint[mm]{1}$ permitiendo simplificar
          los cálculos.
\end{itemize}

De esta manera, se obtiene un triángulo que cumple con las siguientes dimensiones
(figura \ref{fig:unitary_triangle}):

\begin{figure}[H]
    \centering
    \includegraphics[width=.3\linewidth]{pictures/ik_unitary_triangle.png}
    \caption{Triángulo con uno de los lados unitario para el \pArm{}.}
    \label{fig:unitary_triangle}
\end{figure}

En particular, el triángulo anterior se puede colocar a modo de
referencia\footnote{las longitudes del triángulo no son equivalentes a las mostradas en el dibujo.}
sobre el brazo tal como se muestra en la imagen \ref{fig:u_triangle_over_arm}:

\begin{figure}[H]
    \centering
    \includegraphics[width=.6\linewidth]{pictures/ik_triangle_over_arm.png}
    \caption{El triángulo con un lado unitario colocado a modo de referencia sobre el \pArm{}.}
    \label{fig:u_triangle_over_arm}
\end{figure}

Para el triángulo mostrado en la figura \ref{fig:unitary_triangle} se aplica
dos veces el teorema del coseno (ecuación \ref{eq:cosine_law_angle}) para obtener
los valores de $\theta_1'$ (ver ecuación \ref{eq:t1}) y $\theta_2'$ (ver ecuación \ref{eq:t2}):

\begin{align}
    \theta_1' & = \arccos{\left(\frac{x'^2 + z'^2 + \left(\frac{\overline{A_U}}{\overline{A_L}}\right)^2 - 1} %
    {2\sqrt{x'^2 + z'^2} \cdot \frac{\overline{A_U}}{\overline{A_L}}}\right)} \label{eq:t1}                   \\[2ex]
    \theta_2' & = \arccos{\left(\frac{x'^2 + z'^2 + 1 - \left(\frac{\overline{A_U}}{\overline{A_L}}\right)^2} %
    {2\sqrt{x'^2 + z'^2}}\right)} \label{eq:t2}
\end{align}

Como se puede apreciar en las ecuaciones anteriores, todavía no se tienen los valores finales
de los ángulos sino una primera aproximación a ellos. Esto es debido a que tanto el ángulo
$\theta_1$ como el ángulo $\theta_2$ que aparecen en el triángulo de la figura \ref{fig:unitary_triangle}
no se encuentran paralelos al plano del suelo, en este caso, el eje $X$ (tal y como se puede ver
en la figura \ref{fig:u_triangle_over_arm}).

Para poder obtener los ángulos reales se ha de añadir el ángulo `$\phi$' que relaciona el triángulo
\ref{fig:u_triangle_over_arm} con el plano del suelo, tal y como se puede ver en la figura
\ref{fig:final_triangle}:

\begin{figure}[H]
    \centering
    \includegraphics[width=.5\linewidth]{pictures/ik_final.png}
    \caption{Triángulo final orientativo junto con el ángulo $\phi$ respecto al plano del suelo.}
    \label{fig:final_triangle}
\end{figure}

La obtención de dicho ángulo se muestra en la ecuación \ref{eq:phi}

\begin{equation}\label{eq:phi}
    \phi = \arctan{\frac{x'}{z'}}~rad
\end{equation}

y los ángulos finales $\theta_1$ y $\theta_2$ se obtienen mediante la ecuación
\ref{eq:final_theta}\footnote{el ángulo $\phi$ se suma o se resta según la orientación
    del triángulo ya que la arcotangente varía en signo según la posición de los puntos
    $x'$ y $z'$, tal como se muestra en la documentación al desarrollador\cite{UArmDeveloperSwiftProForArduino}.}:

\begin{equation}\label{eq:final_theta}
    \left\{
    \begin{aligned}
        \theta_1 & = \theta_1' - \phi \\
        \theta_2 & = \theta_2' + \phi \\
    \end{aligned}
    \right.
\end{equation}

Finalmente, por seguridad, se puede comprobar que en efecto los distintos ángulos obtenidos
están comprendidos dentro del rango de movimiento de cada una de las articulaciones
\cite{UArmDeveloperSwiftProForArduino}:

\begin{align*}
    \theta_0 & \in \left[\frac{\pi}{18}, \frac{151}{180}\pi\right] \\
    \theta_1 & \in \left[0, \frac{113}{150}\pi\right]              \\
    \theta_2 & \in \left[0, \frac{1199}{1800}\pi\right]
\end{align*}