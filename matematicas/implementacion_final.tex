Una vez concluido el estudio sobre el fundamento matemático del proyecto, se decidió
qué usar de lo visto anteriormente.

Por una parte, se vio cómo el uso de la fuerza bruta generando un mapa de ángulos
y puntos resultaba inviable teniendo en cuenta el espacio disponible en el microcontrolador
así como el tiempo necesario para calcularlo.

Con respecto a las funciones Jacobianas, si bien su estudio permite crear muchas
relaciones entre velocidades y fuerzas, dado que la masa del robot es pequeña y la
velocidad es constante, el uso de dichas funciones para el control del mismo no añade
mucha más información de la que ya se dispone. Además, en favor de lo anterior, en el
código fuente original del $\mu$Arm tampoco contempla las funciones Jacobianas a la hora
de manejar ni los puntos ni la velocidad \cite{UArmDeveloperSwiftProForArduino}, por lo
que se puede asumir que su uso no es necesario.

Por esto mismo, el control del brazo se realizará utilizando la cinemática directa
para obtener el punto $\left\{x, y, z\right\}$ del \textit{end--effector} cuando se
aplican unos ángulos de entrada $\left\{\theta_0, \theta_1, \theta_2\right\}$; y la
cinemática inversa para obtener la relación entre un punto de entrada $\left\{x, y, z\right\}$
y los ángulos que lo generan $\left\{\theta_0, \theta_1, \theta_2\right\}$.

Teniendo en cuenta las características mencionadas sobre el microcontrolador a utilizar,
el tiempo de ambas operaciones es bastante pequeño (unos $\numprint[\tcmu s]{15}$ la
cinemática directa y $\numprint[\tcmu s]{100}$ para la cinemática inversa, según una
estimación con el simulador.), por lo que su uso no añade un desfase suficientemente
grande como para considerarse notorio.