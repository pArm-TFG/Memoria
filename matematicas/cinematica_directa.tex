La cinemática directa permite conocer la rotación relativa entre dos articulaciones
junto con la traslación relativa entre las mismas, utilizando para ello la matriz
definida en la ecuación \ref{eq:dh-matrix}.

Como se puede ver en dicha matriz, se relaciona una articulación $i$ con el equivalente
anterior, en este caso $i - 1$, obteniéndose así la rotación relativa $R'$ y la
traslación relativa $T'$. Si bien esta aproximación es sencilla, solo permite relacionar
dos articulaciones entre sí y que estén en principio unidas por la normal común.

En este brazo se disponen de tres articulaciones, como se muestra en la tabla
\ref{tab:dh-params}, por lo que interesa obtener la matriz $^{0}T_{3}$, la cual relaciona
directamente todas las articulaciones del brazo y permite obtener la rotación absoluta
$R$ y la traslación absoluta $T$.

La obtención de esta matriz es trivial y responde a la ecuación \ref{eq:t03}:

\begin{equation}
    \label{eq:t03}
    ^0T_3 = {^0T_1} \cdot {^1T_2} \cdot {^2T_3}
\end{equation}

Dada que la multiplicación de matrices ha de realizarse en cierto orden, los
factores han de permanecer en la misma posición siempre. De esta manera, se obtienen
las siguientes matrices intermedias (ecuaciones \ref{eq:t01}, \ref{eq:t12}, \ref{eq:t23})
y la matriz final (ecuación \ref{eq:t03-final}):

\begin{align}
    ^0T_1 & =
    \begin{bmatrix}
        \cos{\theta_{1}} & 0 & \sin{\theta_{1}}   & d_{1} \cos{\theta_{1}} \\
        \sin{\theta_{1}} & 0 & - \cos{\theta_{1}} & d_{1} \sin{\theta_{1}} \\
        0                & 1 & 0                  & a_{1}                  \\
        0                & 0 & 0                  & 1                      \\
    \end{bmatrix}\label{eq:t01} \\[1ex]
    ^1T_2 & =
    \begin{bmatrix}
        \cos{\theta_{2}} & \sin{\theta_{2}}   & 0  & a_{2} \cos{\theta_{2}} \\
        \sin{\theta_{2}} & - \cos{\theta_{2}} & 0  & a_{2} \sin{\theta_{2}} \\
        0                & 0                  & -1 & 0                      \\
        0                & 0                  & 0  & 1                      \\
    \end{bmatrix}\label{eq:t12} \\[1ex]
    ^2T_3 & =
    \begin{bmatrix}
        \cos{\theta_{3}} & - \sin{\theta_{3}} & 0 & a_{3} \cos{\theta_{3}} \\
        \sin{\theta_{3}} & \cos{\theta_{3}}   & 0 & a_{3} \sin{\theta_{3}} \\
        0                & 0                  & 1 & 0                      \\
        0                & 0                  & 0 & 1                      \\
    \end{bmatrix}\label{eq:t23} \\[2ex]
    ^3T_0 & =
    {\small\begin{bmatrix}
        \cos{\theta_{1}} \cos{\theta_{2} - \theta_{3}} & \sin{\theta_{2} - \theta_{3}} \cos{\theta_{1}}                                                                    & - \sin{\theta_{1}} \\
        \sin{\theta_{1}} \cos{\theta_{2} - \theta_{3}} & \sin{\theta_{1}} \sin{\theta_{2} - \theta_{3}}                                                                    & \cos{\theta_{1}}   \\
        \sin{\theta_{2} - \theta_{3}}                  & - \cos{\theta_{2} - \theta_{3}}                                                                                   & 0                  \\
        0                                              & 0                                                                                                                 & 0                  \\
                                                       & \qquad T_{X} + \left(a_{2} \cos{\theta_{2}} + a_{3} \cos{\theta_{2} - \theta_{3}} + d_{1}\right) \cos{\theta_{1}}                      \\
                                                       & \qquad \left(a_{2} \cos{\theta_{2}} + a_{3} \cos{\theta_{2} - \theta_{3}} + d_{1}\right) \sin{\theta_{1}}                              \\
                                                       & \qquad T_{Z} + a_{1} + a_{2} \sin{\theta_{2}} + a_{3} \sin{\theta_{2} - \theta_{3}}                                                    \\
                                                       & \qquad 1                                                                                                                               \\
    \end{bmatrix}}\label{eq:t03-final}
\end{align}

Como se puede apreciar en la matriz \ref{eq:t03-final}, se muestra un añadido a los valores
de la traslación $T$: $T_X$ y $T_Z$. Estas dos variables representan traslaciones tanto 
en el eje $X$ y como en el eje $Z$, las cuales aparecen debido a que en el brazo 
se contemplan variaciones en la longitud de ciertos segmentos, que no se han tenido en 
cuenta a la hora de definir los parámetros de \textit{Denavit--Hartenberg} y que, 
en el momento de la obtención de las matrices, no afectan directamente a los cálculos 
(se anulan con los senos y cosenos) pero han de aparecer en la ``matriz final'' ,
permitiendo así la obtención precisa de la posición del \textit{end--effector}. En particular, 
la traslación $T_Z$ añade la longitud del segmento $a_4$ que fue ignorado en los 
parámetros de \textit{Denavit--Hartenberg}, como se mostró en la tabla \ref{tab:dh-params}.

Con estos valores se pueden obtener directamente las ecuaciones que relacionan
los ángulos de entrada $\left\{\theta_1, \theta_2, \theta_3\right\}$ con el punto
final $\left\{x, y, z\right\}$ en el cual se situará el brazo robótico (ecuación
\ref{eq:from_thetas_to_xyz}):

\begin{equation}
    \label{eq:from_thetas_to_xyz}
    \left.\begin{aligned}
        x & = T_{X} + \left(a_{2} \cos{\theta_{2}} + a_{3} \cos{\theta_{2} - \theta_{3}} + d_{1}\right) \cos{\theta_{1}} \\
        y & = \left(a_{2} \cos{\theta_{2}} + a_{3} \cos{\theta_{2} - \theta_{3}} + d_{1}\right) \sin{\theta_{1}}         \\
        z & = T_{Z} + a_{1} + a_{2} \sin{\theta_{2}} + a_{3} \sin{\theta_{2} - \theta_{3}}                               \\
    \end{aligned}
    \right\}
\end{equation}

Si bien estas operaciones se pueden hacer manualmente, los cálculos simbólicos pueden
resultar algo complejos y se han realizado utilizando una librería anteriormente
desarrollada \cite{UPMRoboticsUarm2019b}
(ver código fuente en el anexo \ref{anex:pArm-configurator}) y, además, se ha creado
un \textit{Jupyter Notebook} interactivo para poder realizar la configuración a medida
e ir viendo los pasos que se han ido realizando \cite{PArmTFGPArmconfigurator2020}.
Dicho cuaderno es accesible desde la URL especificada en el anexo \ref{anex:jupyter_binder}.
