Las matrices Jacobianas son una herramienta que permite definir la relación
dinámica entre dos representaciones diferentes de un sistema. Para un manipulador
de $n$ grados de libertad (con $n > 1$) se puede definir la posición del mismo de dos
formas posibles:

\begin{enumerate}
    \item Mediante la posición y orientación del \textit{end--effector}, denominado por $x$.
    \item Mediante el conjunto de los ángulos de las articulaciones, denominado por $q$.
\end{enumerate}

El modo de funcionamiento de las matrices Jacobianas se puede definir como el efecto
que se produce en el \textit{end--effector} `$x$' tras un movimiento de las articulaciones
`$q$', entendiendo así la Jacobiana como la matriz transformada de la velocidad.

Formalmente, la matriz Jacobiana se define como un conjunto de ecuaciones diferenciales
parciales (denotado en la ecuación \ref{eq:jacobian_def}):

\begin{equation}\label{eq:jacobian_def}
    J = \frac{\partial x}{\partial q}
\end{equation}

la cual puede ser expresada como:

\begin{equation}\label{eq:jacobian_x}
    \dot{x} = J \cdot \dot{q}
\end{equation}

donde $\dot{x}$ y $\dot{q}$ representan las derivadas de $x, q$ respecto al tiempo.

En la ecuación \ref{eq:jacobian_x} se expresa que la velocidad del \textit{end--effector}
es igual al producto de la Jacobiana $J$ multiplicada por la velocidad de las articulaciones.
¿Para qué resulta útil tener estos datos? La expresión \ref{eq:jacobian_x} permite el
trabajar con trayectorias en un espacio diferente al que se dispone normalmente\cite{travisdewolfRobotControlPart2013a}.
Esto resulta útil ya que permite el control del \textit{end--effector} mediante la
generación de señales de control (en términos de fuerza) a aplicar en $\left(x, y, z\right)$.
Las matrices Jacobianas pues permiten un cálculo directo de las señales de control en un
espacio controlado, como son los torques de los motores/articulaciones, dada otra señal
que no controlamos, como la fuerza a aplicar en el \textit{end--effector}.

Anteriormente se ha visto que la Jacobiana representa la relación entre velocidades
parciales del \textit{end--effector} y las articulaciones, pero se ha hablado de trabajar
con las fuerzas de cada uno de ellos. Para el \pArm{}, se pueden definir las siguientes
premisas:

\begin{align*}
    x &= \left[x, y, z\right]^T \\
    q &= \left[\theta_0, \theta_1, \theta_2\right]^T
\end{align*}

Como se conoce la velocidad, se puede definir el trabajo $\left(W\right)$ como la fuerza
que hay que aplicar durante una distancia, definido por la ecuación \ref{eq:work}. Por
otra parte, la potencia $\left(P\right)$ se define como la cantidad de trabajo efectuado
por unidad de tiempo\cite{PotenciaFisica2020}, definido por la ecuación \ref{eq:power}.

\begin{align}
    W &= \int{F^T \cdot v~dt} \label{eq:work} \\[1ex]
    P &= \frac{W}{\varDelta t} \label{eq:power}
\end{align}

Por el teorema de la conservación de la energía el trabajo se realiza a la misma
velocidad indiferentemente de la caracterización del sistema, por lo que las ecuaciones
anteriores se pueden escribir tanto en términos del \textit{end--effector}
como en términos de las articulaciones (ecuación \ref{eq:p_eq}):

\begin{equation}\label{eq:p_eq}
    P = \left\{\begin{aligned}
        F^T_x & \cdot \dot{x} \\
        F^T_q & \cdot \dot{q} 
    \end{aligned}\right.
\end{equation}

donde se tiene que:

\begin{equation*}
    \left\{\begin{aligned}
        F_x & \equiv \text{fuerza aplicada al brazo} \\
        F_q & \equiv \text{fuerza aplicada a las articulaciones} \\
        \dot{x} & \equiv \text{velocidad del \textit{end--effector}} \\
        \dot{q} & \equiv \text{velocidad angular de las articulaciones}
    \end{aligned}\right.
\end{equation*}

Igualando las ecuaciones anteriores, obtenemos:

\begin{align*}
    F^T_{q_{hand}} \cdot{} \dot{q} &= F^T_x \dot{x} \\
    F^T_{q_{hand}} \cdot{} \dot{q} &= F^T_x \cdot{} J_{ee}\left(q\right) \cdot \dot{q} \\
    F^T_{q_{hand}} &= J^T_{ee}\left(q\right) \cdot F_x
\end{align*}

donde $J_{ee}\left(q\right)$ es la matriz Jacobiana para el \textit{end--effector}
del robot y $F_{q_{hand}}$ representa las fuerzas en el espacio de articulaciones
que afecta al movimiento del brazo.

De esta manera, se puede utilizar la matriz Jacobiana no solamente para relacionar
la velocidad de un espacio de estados con otro sino que además se puede utilizar
para calcular las fuerzas que se necesiten en el espacio de articulaciones para conseguir
las fuerzas deseadas en el espacio del \textit{end--effector}.

\subsection*{Construyendo la matriz Jacobiana}
Como se mostró anteriormente, la velocidad del \textit{end--effector} se puede
expresar como el producto de la matriz Jacobiana por la velocidad de las articulaciones
(ecuación \ref{eq:jacobian_x}). Para dicha ecuación se tienen los siguientes datos:

\begin{equation*}
    \dot{x} = J \cdot \dot{q}
    \left\{\begin{aligned}
        \dot{x} &= \left(X_e, Y_e, Z_e, \phi_e\right) \\
        \dot{q} &= \left(\theta_0, \cdots, \theta_n,~d_1, \cdots, d_n\right) \\
        J &= \text{matriz Jacobiana}
    \end{aligned}\right.
\end{equation*}

La obtención de la matriz Jacobiana $J\left(\dot{q}\right)$ se ha de realizar obteniendo
las submatrices Jacobianas que relacionan la velocidad lineal `$v$' y la velocidad angular
`$\omega$'. La matriz Jacobiana que relaciona la velocidad lineal se define como
(ecuación \ref{eq:j_v}):

\begin{equation}\label{eq:j_v}
    J_v\left(\dot{q}\right) = 
    \begin{bmatrix}
        \frac{\partial X_e}{\partial \theta_0} & \frac{\partial X_e}{\theta_1} & \cdots & \frac{\partial X_e}{\theta_n} & \frac{\partial X_e}{d_1} & \cdots & \frac{\partial X_e}{d_n} \\[3ex]
        \frac{\partial Y_e}{\partial \theta_0} & \frac{\partial Y_e}{\theta_1} & \cdots & \frac{\partial Y_e}{\theta_n} & \frac{\partial Y_e}{d_1} & \cdots & \frac{\partial Y_e}{d_n} \\[3ex]
        \frac{\partial Z_e}{\partial \theta_0} & \frac{\partial Z_e}{\theta_1} & \cdots & \frac{\partial Z_e}{\theta_n} & \frac{\partial Z_e}{d_1} & \cdots & \frac{\partial Z_e}{d_n} \\
    \end{bmatrix}
\end{equation}

y la matriz que relaciona la velocidad angular se define como (ecuación \ref{eq:j_w}):

\begin{equation}\label{eq:j_w}
    J_{\omega}\left(\dot{q}\right) =
    \begin{bmatrix}
        \frac{\partial \phi_X}{\partial \theta_0} & \frac{\partial \phi_X}{\theta_1} & \cdots & \frac{\partial \phi_X}{\theta_n} & \frac{\partial \phi_X}{d_1} & \cdots & \frac{\partial \phi_X}{d_n} \\[3ex]
        \frac{\partial \phi_Y}{\partial \theta_0} & \frac{\partial \phi_Y}{\theta_1} & \cdots & \frac{\partial \phi_Y}{\theta_n} & \frac{\partial \phi_Y}{d_1} & \cdots & \frac{\partial \phi_Y}{d_n} \\[3ex]
        \frac{\partial \phi_Z}{\partial \theta_0} & \frac{\partial \phi_Z}{\theta_1} & \cdots & \frac{\partial \phi_Z}{\theta_n} & \frac{\partial \phi_Z}{d_1} & \cdots & \frac{\partial \phi_Z}{d_n} \\
    \end{bmatrix}
\end{equation}

De esta manera, la matriz Jacobiana `$J$' se puede definir como 
(ecuación \ref{eq:j})\footnote{el cálculo de las matrices Jacobianas puede resultar
complejo de realizar sobre todo a nivel simbólico, por lo que se deja en el anexo
\ref{anex:jupyter_binder} un enlace a un \textit{Jupyter Notebook} que agiliza y guía
durante el proceso de obtención de estas matrices. El código fuente para su obtención
no obstante se encuentra disponible en el anexo \ref{lst:manipulator_py}.}:

\begin{gather}\label{eq:j}
        J_{ee}\left(\dot{q}\right) = 
        \begin{bmatrix}
            J_v\left(\dot{q}\right) \\
            J_{\omega}\left(\dot{q}\right)
        \end{bmatrix} = \\[2ex]
        {\footnotesize\begin{bmatrix}
            - \left(a_{2} \cos{\theta_{1}} + a_{3} \cos{\theta_{1} - \theta_{2}} + d_{1}\right) \sin{\theta_{0}} & \left(- a_{2} \sin{\theta_{1}} - a_{3} \sin{\theta_{1} - \theta_{2}}\right) \cos{\theta_{0}} & a_{3} \sin{\theta_{1} - \theta_{2}} \cos{\theta_{0}} \\
            \left(a_{2} \cos{\theta_{1}} + a_{3} \cos{\theta_{1} - \theta_{2}} + d_{1}\right) \cos{\theta_{0}}   & \left(- a_{2} \sin{\theta_{1}} - a_{3} \sin{\theta_{1} - \theta_{2}}\right) \sin{\theta_{0}} & a_{3} \sin{\theta_{0}} \sin{\theta_{1} - \theta_{2}} \\
            0                                                                                                                                              & a_{2} \cos{\theta_{1}} + a_{3} \cos{\theta_{1} - \theta_{2}}                                               & - a_{3} \cos{\theta_{1} - \theta_{2}}                              \\[1ex]
            0                                                                                                                                              & 1                                                                                                                                      & -1                                                                               \\
            0                                                                                                                                              & 0                                                                                                                                      & 0                                                                                \\
            1                                                                                                                                              & 0                                                                                                                                      & 0                                                                                \\
        \end{bmatrix}} \nonumber
\end{gather}

Una de las utilidades de la matriz Jacobiana es la obtención de los puntos críticos,
es decir, aquellos en los que el determinante de dicha matriz se hace cero. Los puntos
críticos resultan de especial interés ya que definen posiciones en el manipulador que o
bien son inalcanzables o bien someten a la estructura física del mismo a una gran tensión,
pudiendo resultar dañado en el proceso o de llegar a una ``posición de no retorno'', donde
los motores que componen el brazo puede que no dispongan de suficiente fuerza para moverse a otra
posición.

Para la matriz Jacobiana anterior (ecuación \ref{eq:j}), se obtiene el siguiente 
determinante (ecuación \ref{eq:j_det}):

\begin{equation}\label{eq:j_det}
    \left|J_{ee}\left(\dot{q}\right)\right| = - a_{2} a_{3} \left(a_{2} \cos{\theta_{1}} + a_{3} \cos{\theta_{1} - \theta_{2}} + d_{1}\right) \sin{\theta_{2}}
\end{equation}

Analíticamente se puede observar que los puntos críticos del \pArm{} se dan para 
los valores de $\theta_2 = 0$ y $\theta_2 = \pi$, punto en el que el brazo está o bien
completamente recogido o bien completamente estirado. La cuestión radica en que, viendo
la configuración geométrica del brazo robótico, el ángulo de $\pi~rad$ se
vuelve inalcanzable ya que los valores máximos del ángulo $\theta_2$ son \cite{UArmDeveloperSwiftProForArduino}:

\begin{equation*}
    \theta_2 \in \left[0, \frac{1199}{1800}\pi\right]
\end{equation*}

Por el contrario, la posición de $0~rad$ se habrá de tener en cuenta para
evitar que el brazo esté expuesto a un nivel elevado de tensión durante tiempo prolongado.
Entre los dos segmentos superiores del brazo robótico se situará un fin de carrera a
efectos de evitar dicha tensión además de regular y calibrar los motores.

Al igual que en el caso de la cinemática directa, se puede obtener una matriz Jacobiana
inversa que permite, dada una fuerza en un otro espacio $\dot{x}$ obtener qué par han
de generar las articulaciones $\dot{q}$ para obtener dicha fuerza. La Jacobiana inversa
depende directamente de que el determinante sea distinto de cero ya que, en otro caso,
implicará que la matriz es una matriz singular y que por consiguiente no es invertible
\cite{InvertibleMatrix2020}.

Para el caso anterior existe una ``pseudo--inversa'' de Moore--Penrose \cite{PseudoinversaMoorePenrose2020}
la cual permite la obtención de una matriz inversa aún cuando su determinante es cero.
Dicha pseudo--inversa se denota $J^+$ y se define por (ecuación \ref{eq:j+}):

\begin{equation}\label{eq:j+}
    J^+ = J^T (J \cdot J^T)^{-1}
\end{equation}

Además, se cumple que si la inversa de la matriz Jacobiana existe entonces su pseudo-inversa
también existirá, y será igual a la matriz inversa:

\begin{equation*}
    J^+ = J^{-1} \iff \exists J^{-1}
\end{equation*}

Como los puntos críticos son $\theta_2 = 0$ y $\theta_2 = \pi$ se puede obtener un valor
de la inversa que será igual a la pseudo--inversa, donde ambas dependen del parámetro
$\theta_2$ para existir. El valor que se obtiene de la inversa es el siguiente (ecuación
\ref{eq:pinv})\footnote{el cálculo simbólico de tanto la inversa como de la pseudo--inversa
puede resultar algo complejo por lo que se ha dejado en el anexo \ref{anex:jupyter_binder}
un \textit{Jupyter Notebook} para realizar las operaciones de forma interactiva y guiada.
No obstante, el código fuente se encuentra disponible en el anexo \ref{lst:manipulator_py}.}:

\begin{gather}\label{eq:pinv}
    J^{-1} = J^+ = \\
    {\footnotesize\begin{bmatrix}
        - \frac{\sin{\theta_{0}}}{a_{2} \cos{\theta_{1}} + a_{3} \cos{\theta_{1} - \theta_{2}} + d_{1}}                                                 & \frac{\cos{\theta_{0}}}{a_{2} \cos{\theta_{1}} + a_{3} \cos{\theta_{1} - \theta_{2}} + d_{1}}                                                   & 0                                                                                                                                             \\
        - \frac{\cos{\theta_{0}} \cos{\theta_{1} - \theta_{2}}}{a_{2} \sin{\theta_{2}}}                                                                 & - \frac{\sin{\theta_{0}} \cos{\theta_{1} - \theta_{2}}}{a_{2} \sin{\theta_{2}}}                                                                 & - \frac{\sin{\theta_{1} - \theta_{2}}}{a_{2} \sin{\theta_{2}}}                                                    \\
        - \frac{\left(a_{2} \cos{\theta_{1}} + a_{3} \cos{\theta_{1} - \theta_{2}}\right) \cos{\theta_{0}}}{a_{2} a_{3} \sin{\theta_{2}}} & - \frac{\left(a_{2} \cos{\theta_{1}} + a_{3} \cos{\theta_{1} - \theta_{2}}\right) \sin{\theta_{0}}}{a_{2} a_{3} \sin{\theta_{2}}} & - \frac{a_{2} \sin{\theta_{1}} + a_{3} \sin{\theta_{1} - \theta_{2}}}{a_{2} a_{3} \sin{\theta_{2}}} \\
    \end{bmatrix}} \nonumber
\end{gather}