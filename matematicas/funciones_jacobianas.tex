Las matrices Jacobianas son una herramienta que permite definir la relación
dinámica entre dos representaciones diferentes de un sistema. Para un manipulador
de $n$ grados de libertad (con $n > 1$) se puede definir la posición del mismo de dos
formas posibles:

\begin{enumerate}
    \item Mediante la posición y orientación del \textit{end--effector}, denominado por $x$.
    \item Mediante el conjunto de los ángulos de las articulaciones, denominado por $q$.
\end{enumerate}

El modo de funcionamiento de las matrices Jacobianas se puede definir como el efecto
que se produce en el \textit{end--effector} `$x$' tras un movimiento de las articulaciones
`$q$', entendiendo así la Jacobiana como la matriz transformada de la velocidad.

Formalmente, la matriz Jacobiana se define como un conjunto de ecuaciones diferenciales
parciales (denotado en la ecuación \ref{eq:jacobian_def}):

\begin{equation}\label{eq:jacobian_def}
    J = \frac{\partial x}{\partial q}
\end{equation}

la cual puede ser expresada como:

\begin{equation}\label{eq:jacobian_x}
    \dot{x} = J \cdot \dot{q}
\end{equation}

donde $\dot{x}$ y $\dot{q}$ representan las derivadas de $x, q$ respecto al tiempo.

En la ecuación \ref{eq:jacobian_x} se expresa que la velocidad del \textit{end--effector}
es igual al producto de la Jacobiana $J$ multiplicada por la velocidad de las articulaciones.
¿Para qué resulta útil tener estos datos? La expresión \ref{eq:jacobian_x} permite el
trabajar con trayectorias en un espacio diferente al que se dispone normalmente\cite{travisdewolfRobotControlPart2013a}.
Esto resulta útil ya que permite el control del \textit{end--effector} mediante la
generación de señales de control (en términos de fuerza) a aplicar en $\left(x, y, z\right)$.
Las matrices Jacobianas pues permiten un cálculo directo de las señales de control en un
espacio controlado, como son los torques de los motores/articulaciones, dada otra señal
que no controlamos, como la fuerza a aplicar en el \textit{end--effector}.

Anteriormente se ha visto que la Jacobiana representa la relación entre velocidades
parciales del \textit{end--effector} y las articulaciones, pero se ha hablado de trabajar
con las fuerzas de cada uno de ellos. Para el \pArm{}, se pueden definir las siguientes
premisas:

\begin{align*}
    x &= \left[x, y, z\right]^T \\
    q &= \left[q_0, q_1, q_2\right]^T
\end{align*}

Como se conoce la velocidad, se puede definir el trabajo $\left(W\right)$ como la fuerza
que hay que aplicar durante una distancia, definido por la ecuación \ref{eq:work}. Por
otra parte, la potencia $\left(P\right)$ se define como la cantidad de trabajo efectuado
por unidad de tiempo\cite{PotenciaFisica2020}, definido por la ecuación \ref{eq:power}.

\begin{align}
    W &= \int{F^T \cdot v~dt} \label{eq:work} \\[1ex]
    P &= \frac{W}{\varDelta t} \label{eq:power}
\end{align}

Por el teorema de la conservación de la energía el trabajo se realiza a la misma
velocidad indiferentemente de la caracterización del sistema, por lo que las ecuaciones
anteriores se pueden escribir tanto en términos del \textit{end--effector}
como en términos de las articulaciones (ecuación \ref{eq:p_eq}):

\begin{equation}\label{eq:p_eq}
    P = \left\{\begin{aligned}
        F^T_x & \cdot \dot{x} \\
        F^T_q & \cdot \dot{q} 
    \end{aligned}\right.
\end{equation}

donde se tiene que:

\begin{equation*}
    \left\{\begin{aligned}
        F_x & \equiv \text{fuerza aplicada al brazo} \\
        F_q & \equiv \text{fuerza aplicada a las articulaciones} \\
        \dot{x} & \equiv \text{velocidad del \textit{end--effector}} \\
        \dot{q} & \equiv \text{velocidad angular de las articulaciones}
    \end{aligned}\right.
\end{equation*}

Igualando las ecuaciones anteriores, obtenemos:

\begin{align*}
    F^T_{q_{hand}} \cdot{} \dot{q} &= F^T_x \dot{x} \\
    F^T_{q_{hand}} \cdot{} \dot{q} &= F^T_x \cdot{} J_{ee}\left(q\right) \cdot \dot{q} \\
    F^T_{q_{hand}} &= J^T_{ee}\left(q\right) \cdot F_x
\end{align*}

donde $J_{ee}\left(q\right)$ es la matriz Jacobiana para el \textit{end--effector}
del robot y $F_{q_{hand}}$ representa las fuerzas en el espacio de articulaciones
que afecta al movimiento del brazo.

De esta manera, se puede utilizar la matriz Jacobiana no solamente para relacionar
la velocidad de un espacio de estados con otro sino que además se puede utilizar
para calcular las fuerzas que se necesiten en el espacio de articulaciones para conseguir
las fuerzas deseadas en el espacio del \textit{end--effector}.