El desarrollo de este proyecto tiene un gran peso matemático: el movimiento de los
motores junto con la actuación conjunta de los mismos permite definir y alcanzar
diversas posiciones a lo largo del rango de movilidad del brazo.

La relación entre los ángulos y el punto final no es trivial y es necesario un estudio
previo de distintos factores para poder hacerlo correctamente. Por una parte,
es necesario definir de la manera más precisa posible la configuración geométrica del
brazo. Dicha configuración relaciona los distintos segmentos que
conforman el manipulador, en particular las posibles articulaciones que
los componen y que se denotan por $Z_i$, donde $i$ es el número de la articulación.

En particular, dichas relaciones son:

\begin{itemize}
    \item El ángulo presente entre dos articulaciones adyacentes
          $\widehat{Z_aZ_b}~rad$, el cual se denota por $\alpha_b$.
    \item La distancia presente entre dos articulaciones adyacentes
          $\overline{Z_aZ_b}$, representada por $a_b$.
    \item El sentido de la rotación de una articulación,
          $\overrightarrow{X_aY_a}$, denotada por $\theta_a$.
\end{itemize}

Estas relaciones permiten establecer la configuración geométrica del robot,
fundamental para poder definir los movimientos posibles del robot y generar tanto
las matrices de la cinemática directa como obtener las ecuaciones de la cinemática
inversa. Además, se puede obtener de la misma manera las matrices Jacobianas que
permiten obtener datos útiles como el trabajo, la velocidad o la potencia.

Para el $\mu$Arm, se obtuvieron las siguientes configuraciones geométricas:

\begin{figure}[H]
    \centering
    \begin{minipage}{.4\linewidth}
        \centering
        \includegraphics[width=\textwidth]{pictures/geometric_configuration_2.png}
        \caption{Configuración geométrica del $\mu$Arm.}
        \label{fig:uArm_gc}
    \end{minipage}
    \hfill
    \begin{minipage}{.48\linewidth}
        \centering
        \includegraphics[width=\textwidth]{pictures/axis.png}
        \caption{Los distintos grados de libertad del $\mu$Arm, representados por $Z_i$.}
        \label{fig:uArm_axis}
    \end{minipage}
\end{figure}

Con estos valores, ya se pueden obtener las distancias entre articulaciones así
como las desviaciones entre las mismas, si las hay. En el caso particular del $\mu$Arm,
se obtiene unos datos como los siguientes (las medidas se han obtenido desde la guía del
desarrollador de UFACTORY\cite{ufactoryUArmSwiftPro2017}):

\begin{table}[ht]
    \begin{minipage}{.49\linewidth}
        \begin{figure}[H]
            \centering
            \includegraphics[width=\linewidth]{pictures/sizes.png}
            \caption{Longitudes del brazo robótico \cite{ufactoryUArmSwiftPro2017}.}
            \label{fig:sizes}
        \end{figure}
    \end{minipage}
    \hfill
    \begin{minipage}{.49\linewidth}
        \centering
        \begin{tabular}{|| c | c c ||}
            \hline
            $i$ & $a_i~(mm.)$ & $d_i~(mm.)$ \\ [0.5ex]
            \hline\hline
            $1$ & $13.2$      & $106.1$     \\
            \hline
            $2$ & $142$       & $0$         \\
            \hline
            $3$ & $158.8$     & $0$         \\
            \hline
            $4$ & $44.5$      & $0$         \\ [1ex]
            \hline
        \end{tabular}
        \caption{Longitudes y desviaciones del manipulador $\mu$Arm.}
    \end{minipage}
\end{table}

Esta información permite construir una tabla de \textit{Denavit--Hartenberg} que
recoge la información del robot. Dicha tabla se conoce también como ``parámetros de
\textit{Denavit--Hartenberg}'', que conforman cuatro variables que recogen, en una
convención particular, la referencia de una cadena cinemática (objetos rígidos unidos
a articulaciones que responden a una función matemática) o de un brazo robótico \cite{DenavitHartenbergParameters2020}.

La convención de parámetros de \textit{Denavit--Hartenberg} son los siguientes:
\begin{itemize}
    \item $d$ -- desviación a lo largo del eje $Z$ respecto a la normal común\footnote{la normal
              común de dos articulaciones que no intersecan se define como la línea perpendicular a
              ambos ejes, que se usa comúnmente para conocer la separación entre ambas dos \cite{CommonNormalRobotics2017}.}
          con el ángulo anterior.
    \item $\theta$ -- el ángulo de rotación desde $x_i$ hasta $y_i$. Sirve para definir
          el sentido de la misma además de indicar el movimiento que realiza dicha articulación.
    \item $a$ -- la distancia de la normal común $\left(\overline{Z_{i - 1}Z_i}\right)$.
    \item $\alpha$ -- el ángulo sobre la normal común, en este caso, de $Z_{i - 1}$ hacia $Z_i$ $\left(\widehat{Z_aZ_b}\right)$.
\end{itemize}

Esta convención es especialmente interesante porque permite definir de forma precisa
las relaciones entre las articulaciones, pudiendo conocer la rotación relativa entre
dos de ellas y la traslación entre sus puntos. Además, aplicando las propiedades
de las matrices, se puede obtener la relación absoluta entre las rotaciones y las
traslaciones, lo que se traduce en conocer el punto exacto $\left\{x, y, z\right\}$ en el que se encuentra el
\textit{end--effector} cuando se giran las articulaciones
$\left\{\theta_0, \theta_1, \cdots, \theta_i\right\}~rad$ respectivamente.

Esta relación se representa mediante una matriz, definida en la ecuación \ref{eq:dh-table}:

\begin{equation}
    {
    \displaystyle \operatorname {}
    ^{i-1}T_{i}=\left[{
                \begin{array}{ccc|c}
                    \cos{\theta_{i}} & -\sin{\theta_{i}}\cos{\alpha_{i}} & \sin{\theta_{i}}\sin{\alpha_{i}}  & d_{i}\cos{\theta_{i}} \\
                    \sin{\theta_{i}} & \cos{\theta_{i}}\cos{\alpha_{i}}  & -\cos{\theta_{i}}\sin{\alpha_{i}} & d_{i}\sin{\theta_{i}} \\
                    0                & \sin{\alpha_{i}}                  & \cos{\alpha_{i}}                  & d_{i}                 \\
                    \hline
                    0                & 0                                 & 0                                 & 1
                \end{array}}\right] =
    \left[{
                \begin{array}{ccc|c}
                      &   &   &   \\
                      & R &   & T \\
                      &   &   &   \\
                    \hline
                    0 & 0 & 0 & 1
                \end{array}}
        \right]
    }
\end{equation}
\label{eq:dh-table}

donde $R$ representa la \textit{rotación relativa} y $T$ la \textit{traslación relativa}.

Otra ventaja de los parámetros de \textit{Denavit--Hartenberg} es la posibilidad de
definir elementos cinemáticos, como la velocidad ($W_{i,j}(k)$) y la aceleración ($H_{i,j}(k)$)
de distintos cuerpos; así como elementos dinámicos, tales como la inercia ($J$),
el momento lineal y angular ($\Gamma$) o las fuerzas y torques aplicados 
($\Phi$)\footnote{si bien estos datos resultan muy útiles para el proyecto no son necesariamente
relevantes, ya que están supeditados a la velocidad y la masa, y el brazo robótico
no presenta ni una masa suficientemente elevada ni alcanza velocidades altas.}.