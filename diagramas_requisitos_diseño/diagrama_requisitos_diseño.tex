Una parte importante de un proyecto integral de ingeniería es la elicitación de requisitos y la creación de diagramas que representen el sistema de manera abstracta en base a dichos requisitos.

El sistema de gobierno del p-Arm esta compuesto de dos subsistema al ser necesaria tanto una placa de control como un ordenador auxiliar desde el cual un operario humano pueda interactuar con el brazo. El software del sistema de control del ordenador será representado mediante un diagrama de clases mientras que el software que ira cargado en la placa de control será representado por un diagrama de bloques general y varios diagramas de estados que detallarán el comportamiento del sistema.

Para realizar dichos diagramas se ha empleado Papyrus, una herramienta de edición gráfica para realización de diagramas. Para modelizar los diagramas del sistema de control que ira en el ordenador auxiliar se ha empleado el estándar UML2 definido por la OMG, por otro lado, para realizar los diagramas del software que será cargado en la placa de control se ha empleado el estándar SysML 1.4 ya que permite mejor representación del sistema empotrado.

A continuación se pueden observar los requisitos que el equipo de desarrollo ha elicitado además de una explicación de los diagramas que representan los dos subsistemas que componen el proyecto,



