Las principales labores desempeñadas por los integrantes del equipo de desarrollo están relacionadas con el ámbito de la ingeniería de computadores, estas labores incluyen, pero no están limitadas a:

\begin{itemize}
    \item Capacidad para sintetizar y proponer una solución basada en un sistema \ac{HW} para dar solución a un problema.
    \item Desarrollo de diagramas que abstraigan la construcción física de una sistema empotrado.
    \item Desarrollo de diagramas que abstraigan el \ac{SW} de un sistema empotrado.
    \item Intervención directa en las labores de fabricación del sistema empotrado
    \item Comprobación del sistema empotrado mediante pruebas \ac{SW} y \ac{HW}.
    \item Documentación del proceso de diseño y fabricación.
    \item Documentación del \ac{SW}
\end{itemize}

Debido a lo anteriormente mencionado, el sueldo propuesto para cada uno de los integrantes, es el sueldo medio de un ingeniero de computadores en España, es decir: 30.798 € / año\footnote{\url{https://www.glassdoor.es/Sueldos/spain-computer-engineer-sueldo-SRCH_IL.0,5_IN219_KO6,23.htm?countryRedirect=true}}

Debido a que los integrantes del equipo de desarrollo han realizado labores ajenas al ámbito de conocimiento de la ingeniería de computadores como son, el diseño y la impresión 3D, el desarrollo de UI/UX, y el desarrollo de un software de alto nivel el equipo ha considerado oportuno el incremento de la supuesta remuneración en 10.000 € al año.

Este incremento dejaría un salario final de 40.798 €/año.