\begin{table}[H]
    \centering
    \begin{tabularx}{\textwidth}{| X | X | c | c | c |}
        \hline
        \textbf{Elemento} & \textbf{Descripción} & \textbf{Cantidad} & \textbf{Precio} & \textbf{Código RS} \\
        \hline
        Servomotor Parallax Inc. & Los servomotores permitirán controlar los distintos movimientos que puede realizar el robot. Será necesario disponer de varios para poder controlar los tres grados de libertad de los que dispondrá el \pArm{}: uno para la base, otro para el primer segmento y un último para el segundo segmento. & 4 & $16,01$ \EUR{} & 781-3058 \\
        \hline
        Rodamiento de bolas NMB, Radial. & La unión entre ejes del brazo robótico se realiza mediante rodamientos, permitiendo así un movimiento fluido y duradero del brazo. & 16 & $5,035$ \EUR{} & 612-6013 \\
        \hline
        Pasador cilíndrico de $20~mm.$ de largo y $4~mm.$ de diámetro & El pasador que permitirá unir dos ejes separados, pasando por medio de los rodamientos anteriores. & 1 bolsa & $7,05$ \EUR{} & 270-439 \\
        \hline
        Microinterruptor de hasta $125V@5A$ & Fin de carrera para controlar la posición del brazo robótico & 4 & $1,709$ \EUR{} & 682-0866 \\
        \hline
        Fungibles & Gastos extras que se estima que puedan surgir (por ejemplo, tornillos, tuercas, otros materiales, etc.) pero cuya cantidad y precio no se conocen a estas alturas del proyecto & -- & $50$ \EUR{} & -- \\
        \hline\hline
        \multicolumn{3}{| l |}{\textbf{Total:}} & \multicolumn{2}{ l |}{$203,32$ \EUR{}} \\
        \hline
    \end{tabularx}
    \caption{Tabla completa de presupuestos.}
    \label{tab:budgets}
\end{table}