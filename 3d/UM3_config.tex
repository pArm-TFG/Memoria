\begin{longtable}{ C{0.5} | C{0.5} | C{1} }
    \caption{Configuración de la Ultimaker 3 para generar material de soporte en \ac{PVA}.}
    \label{tab:um3}
    \endfirsthead
    \endhead
    \hline
    \textbf{Configuración} & \textbf{Valor} & \textbf{Descripción} \\[2ex]
    \hline
    \texttt{Layer height} & $[\numprint[mm]{0.06}$, $\numprint[mm]{0.1}$, $\numprint[mm]{0.15}$, $\numprint[mm]{0.2}]$ & El tamaño de la capa de plástico que es utilizada para crecer verticalmente. Valores pequeños otorgan más precisión pero incrementan el tiempo de impresión. \\
    \hline
    \texttt{Enable Ironing} & \done & Vuelve a recorrer la última capa impresa extruyendo muy poco material para suavizarla al tacto. \\
    \hline
    \texttt{Infill density} & 20\% - 80\% & Cantidad de material introducido en capas huecas. Mayor material incrementa la resistencia pero aumenta el peso y el tiempo de impresión. Se recomienda no usar valores por encima del 50\%. \\
    \hline
    \texttt{Infill pattern} & \textit{Tri--Hexagon} & El patrón de relleno del \textit{infill}. Por defecto son triángulos, pero ``\textit{tri--hexagon}'' gasta más material a cambio de mayor soporte físico. \\
    \hline
    \texttt{Printing temperature} & $\numprint[\tccelsius]{215}$ & Para el \ac{PVA}, se establece la temperatura de impresión en el rango habitual. \\
    \hline
    \texttt{Final printing temperature} & $\numprint[\tccelsius]{200}$ & La temperatura final de acabado de impresión (para el \ac{PVA}). \\
    \hline
    \texttt{Standby temperature} & $\numprint[\tccelsius]{60}$ & La temperatura que se establece en el extrusor cuando no se está utilizando activamente. \\
    \hline
    \texttt{Enable retraction} & \done & Retractar el eje $Z$ cuando el \textit{nozzle} se mueve sobre un área sin objetos impresos. \\
    \hline
    \texttt{Retraction at layer change} & \done & Retractar el \textit{nozzle} cuando se avanza a la siguiente capa de impresión. \\
    \hline
    \texttt{Retraction distance} & $\numprint[mm]{4.5}$ & La cantidad de material que es recogido cuando se retracta el \textit{nozzle}. \\
    \hline
    \texttt{Avoid printed parts when travelling} & \done & Si es posible, evitar mover el cabezal de impresión sobre zonas ya impresas. \\
    \hline
    \texttt{Avoid supports when travelling} & \done & Si es posible, evitar las partes impresas con material soporte. \\
    \hline
    \texttt{Z hop when retracted} & \done & Cuando se mueve la cabeza de impresión, se baja además el \textit{build plate} para evitar colisiones. \\
    \hline
    \texttt{Z hop only over printed parts} & \done & Solo se realiza la retracción anterior si se mueve el cabezal sobre partes ya impresas. \\
    \hline
    \texttt{Z hop after extruder switch} & \done & Retractar el eje $Z$ cuando se cambia de extrusor. \\
    \hline
    \texttt{Enable print cooling} & \done & Activa el ventilador que ayuda a solidificar el plástico recién impreso. \\
    \hline
    \texttt{Regular fan speed} & 20\% & Velocidad a la que suele estar el ventilador. \\
    \hline
    \texttt{Maximum fan speed} & 80\% & Velocidad máxima que puede alcanzar el ventilador. \\
    \hline
    \texttt{Initial fan speed} & 0\% & Empieza con el ventilador apagado. \\
    \hline
    \texttt{Regular fan speed at height} & $\numprint[mm]{0.51}$ & Altura a la cual se activa el ventilador a la velocidad normal. \\
    \hline
    \texttt{Regular fan speed at layer} & 6 & Iniciar, si no se ha alcanzado todavía, los ventiladores a velocidad regular en esa capa de impresión. \\
    \hline
    \texttt{Generate support} & \done & Genera una capa de soporte utilizando el extrusor especificado. \\
    \hline
    \texttt{Support pattern} & \textit{Triangles} o \textit{Tri--Hexagon} & Patrón que utiliza el extrusor para poner material de soporte. \\
    \hline
    \texttt{Support density} & 50\% & La cantidad de soporte que es utilizado para sujetar una pieza. \\
    \hline
    \texttt{Enable support interface} & \done & Se crea una serie de capas con \textit{infill} de casi el 100\% para crear una superficie plana antes de imprimir el nuevo material sobre ella. \\
    \hline
    \texttt{Enable support roof} & \done & Crea la capa mencionada anteriormente sobre el material de soporte, antes de imprimir sobre él. \\
    \hline
    \texttt{Enable support floor} & \done & Crea la capa mencionada anteriormente sobre la cama caliente o la pieza impresa previamente, antes de extruir material de soporte. \\
    \hline
    \texttt{Enable prime tower} & \done & Crea una torre en una esquina de la cama caliente que se utiliza principalmente para limpiar los cabezales después de no haber sido utilizados durante el cambio de material. \\
\end{longtable}