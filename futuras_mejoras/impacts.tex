Si bien es cierto que el $\mu$Arm ya es un sistema avanzado y capaz, como se explicó 
en la Introducción (\ref{ch:intro}), se pretende estudiar y desarrollar un sistema 
propio el cual pueda servir para ayudar y facilitar la entrada a este tipo de 
tecnologías a otras personas, haciéndolo comprensible y, aprovechando la tecnología 
de la impresión en 3D, fabricable por uno mismo.

Además, en relación a los \ac{ODS}, con el desarrollo de este sistema se pretende 
trabajar en:

\begin{itemize}
    \item [4 -] Educación de Calidad\footnote{\url{https://www.un.org/sustainabledevelopment/es/education/}}.
    \item [7 -] Energía Asequible y No Contaminante\footnote{\url{https://www.un.org/sustainabledevelopment/es/energy/}}.
    \item [10 -] Reducción de las desigualdades\footnote{\url{https://www.un.org/sustainabledevelopment/es/inequality/}}.
\end{itemize}

Para el primero, se tiene en cuenta que el producto se desarrollará siguiendo las 
iniciativas \ac{OS} y \ac{OH}, las cuales facilitan el acceso a la información a 
cualquiera que la requiera. %Además, se facilitará el desarrollo al completo, detallado y explicado, con la resolución de los problemas pertinentes y el porqué de ella.

Para el segundo, el \pArm{} utilizará la electricidad como fuente de energía, evitando
así otras más contaminantes como las producidas por combustibles fósiles. En añadido, 
se trabajará para que el consumo de energía sea el menor posible, permitiendo así un 
mayor tiempo de uso con la misma fuente de alimentación y no abusando de los recursos 
de los que se disponen.

Finalmente, se pretende hacer que el \pArm{} tenga un coste bajo, permitiendo así el 
acceso a los recursos y a los procesos de fabricación a todo el mundo que pudiera 
estar interesado. % y que disponga de la cantidad mínima necesaria para poder poner en funcionamiento el brazo robótico.