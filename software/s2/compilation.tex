Durante la compilación es necesario definir opciones adicionales a las básicas
establecidas por defecto, ya que en otro caso la compilación fallará o no contendrá
todas las características esperables.

Por una parte, es necesario definir que se trabaje con \texttt{C99}. Esto se hace
definiendo la opción del compilador adicional \texttt{-std=gnu99}. Además, es recomendable
trabajar con \texttt{double} de 64 bits, pero en otro caso se utilizarán tipos de datos
propios para trabajar con esa precisión:

\lstinputlisting[linerange={43-51}, firstnumber=43, caption=, style=C]{pArm-S2/pArm.X/utils/types.h}

Es recomendable designar al menos $\numprint[KB]{8}$ de memoria RAM dedicada a memoria
\textit{heap} para la creación de espacios de memoria reservados a variables de
forma dinámica. Esto permite que funciones como \lstinline[style=C]{void* malloc(size_t)} o
\lstinline[style=C]{void* realloc(void*, size_t)} no fallen y reserven la cantidad de memoria necesaria.

En lo referente a los modelos de memoria, se pueden establecer todos ellos como
\textit{small}: \texttt{code model}, \texttt{memory model}, \texttt{data model}. Por
otra parte, es recomendable habilitar el nivel de optimización del código \texttt{-O2},
de forma que las librerías auxiliares que se utilizan son optimizadas para reducir
el tamaño y mejorar el rendimiento. Además, se recomienda marcar las siguientes opciones:

\begin{itemize}
    \item ``\textit{Do not override inline}'', para mantener la estructura de las funciones
    \texttt{inline}.
    \item ``\textit{Unroll loops}'', donde se aplica la técnica de optimización de
    bucles en donde, si se conoce de antemano el tamaño del mismo, se desenrolla
    en múltiples instrucciones que son más rápidas en tiempo de ejecución pero
    consumen más memoria de programa.
    \item ``\textit{Align arrays}'', para establecer el modelo de memoria que 
    distribuye las posiciones que componen un \textit{array}.
\end{itemize}

Para reducir el tamaño del código, se recomienda habilitar la opción del \textit{linker} para eliminar
los segmentos de código no utilizados en el programa, con la opción ``\textit{Remove
unused sections}''.

Finalmente, se proveen las siguientes macros del compilador para habilitar distintas
opciones a lo largo del código:

\begin{itemize}
    \item \texttt{PRINTF\_INCLUDE\_CONFIG\_H} -- se usa una configuración propia para
    la librería adaptada de \texttt{printf}.
    \item \texttt{USE\_CUSTOM\_PRINTF} -- se utiliza una versión personalizada de
    \texttt{printf} en lugar de la provista por \lstinline[style=C]{stdio.h}.
    \item \texttt{DEBUG\_ENABLED} -- añade nuevas opciones de depuración al código.
    \item \texttt{CONFIG\_SIMULATOR} -- si se está trabajando con el simulador, para
    deshabilitar opciones no usadas.
    \item \texttt{CLI\_MODE} -- para habilitar el modo de control mediante consola
    de comandos, comunicando directamente desde la \ac{UART}.
\end{itemize}

En la tabla \ref{tab:gcc} se resumen las opciones habilitadas y empleadas:

\LTXtable{\linewidth}{software/s2/compilation_options_table}
