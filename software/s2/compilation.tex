Durante la compilación es necesario definir opciones adicionales a las básicas
establecidas por defecto, ya que en otro caso la compilación fallará o no contendrá
todas las características esperables.

Por una parte, es necesario definir que se trabaje con \texttt{C99}. Esto se hace
definiendo la opción del compilador adicional \texttt{-std=gnu99}. Además, es recomendable
trabajar con \texttt{double} de 64 bits, pero en otro caso se utilizarán tipos de datos
propios para trabajar con esa precisión.

Es recomendable designar al menos $\numprint[KB]{8}$ de memoria RAM dedicada a memoria
\textit{heap} para la creación de espacios de memoria reservados a variables de
forma dinámica.