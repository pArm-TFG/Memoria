El cálculo de los movimientos según los puntos recibidos se realizan utilizando
tanto la cinemática directa como la cinemática inversa, explicadas previamente
en el capítulo \ref{chap:maths}.

Dichas cinemáticas se implementan en el paquete \texttt{kinematics}, implementado
en los listados de códigos \ref{lst:kinematics_h} y \ref{lst:kinematics_c}.

Se definen dos estructuras que permiten manejar los ángulos y los puntos:

\lstinputlisting[linerange={53-73}, firstnumber=53, caption=, style=C]{pArm-S2/pArm.X/utils/types.h}

Con dichos tipos de datos, las funciones dedicadas a la cinemática directa como la
cinemática inversa computan el valor de entrada y actualizan la dirección de memoria
en la que se espera obtener el valor resultante.

En principio, los métodos anteriores suelen funcionar correctamente pero se establecen
unos casos en los cuales la obtención del valor falla:

\begin{itemize}
    \item Cuando alguno de los ángulos provistos/obtenidos es \texttt{NaN}.
    \item Si alguno de los valores provistos en la función $\arccos$ produce un
    resultado mayor a uno, provocando que dicha función falle.
    \item Si alguno de los valores de entrada no se encuentra dentro de los rangos
    de los motores.
\end{itemize}

En todos los casos anteriores, la función falla con un código de error \texttt{EXIT\_FAILURE}.