El \ac{S2} supone una parte fundamental en el desarrollo del proyecto ya que es el
encargado de la gestión al completo del \pArm{}. El \ac{SW} que ejecuta se encuentra
escrito puramente en C (en particular, \texttt{C99}) y se ha programado utilizando
el paradigma de programación estructurada, utilizando subrutinas, secuencias, condiciones
(\texttt{if}, \texttt{switch}) e iteradores (\texttt{for},
\texttt{while})\cite{ProgramacionEstructurada2020}.

La estructura de la aplicación consta de los siguientes paquetes y ficheros:
\begin{itemize}
    \item[\texttt{arm} --] contiene el planificador de movimientos del \pArm{},
    definido por los ficheros \texttt{planner.h} (listado de código \ref{lst:planner_h})
    y \texttt{planner.c} (listado de código \ref{lst:planner_c}).

    \item[\texttt{gcode} --] contiene el intérprete de GCode que se utiliza
    principalmente para la comunicación entre \ac{S1} y \ac{S2}. Se encuentra
    compuesto for los ficheros \texttt{gcode.h} (listado de código
    \ref{lst:gcode_h}) y \texttt{gcode.c} (listado de código \ref{lst:gcode_c}).

    \item[\texttt{motor} --] este paquete aúna la lógica de control de los
    motores que componen el \pArm{}.

    Por una parte, se define un primer nivel de abstracción sobre el control
    de los servomotores implementado en los ficheros \texttt{servo.h} 
    (listado de código \ref{lst:servo_h}) y \texttt{servo.c} 
    (listado de código \ref{lst:servo_c}).

    Sobre lo anterior, se define un segundo nivel de abstracción para el
    control de los servomotores que simplifica las operaciones a realizar sobre
    los mismos. Se encuentra implementado en los ficheros \texttt{motor.h}
    (listado de código \ref{lst:motor_h}) y \texttt{motor.c} (listado de
    código \ref{lst:motor_c}).

    Finalmente, se establece un tercer nivel de abstracción que permite el
    control de los distintos motores mediante la cinemática directa
    (utilizando los ángulos $\left\{\theta_0, \theta_1, \theta_2\right\}$)
    y mediante la cinemática inversa (utilizando los puntos $\left\{x, y, z\right\}$).
    La lógica se encuentra implementada en los ficheros \texttt{kinematics.h}
    (listado de código \ref{lst:kinematics_h}) y \texttt{kinematics.c}
    (listado de código \ref{lst:kinematics_c}).

    \item[\texttt{printf} --] una implementación adaptada para trabajar con la placa
    de control dsPIC33E basada en la librería
    \texttt{mpaland/printf}\footnote{\url{https://github.com/mpaland/printf}}
    \cite{palandMpalandPrintf2020}.

    Se definen nuevos ficheros para la gestión de la librería y se encuentra
    estructurada en \texttt{io.h} (listado de código \ref{lst:io_h}), 
    \texttt{printf\_config.h} (listado de código \ref{lst:printf_config_h}),
    \texttt{printf.h} (listado de código \ref{lst:printf_h}) y \texttt{printf.c}
    (listado de código \ref{lst:printf_c}).

    \item[\texttt{rsa} --] paquete que recoge las principales funcionalidades
    del algoritmo RSA para realizar firma digital y encriptado de datos. Se
    encuentra implementado en los ficheros \texttt{rsa.h} (listado de código
    \ref{lst:rsa_h}) y \texttt{rsa.c} (listado de código \ref{lst:rsa_c}).

    Además, contiene otras funcionalidades útiles como la generación de
    números pseudo--aleatorios, implementada en los ficheros \texttt{rand.h}
    (listado de código \ref{lst:rand_h}) y \texttt{rand.c} (listado de código
    \ref{lst:rand_c}); los algoritmos de \texttt{clz} 
    (``\textit{count leading zeros}'') y \texttt{ctz}
    (``\textit{count trailing zeros}'') implementadas en los ficheros
    \texttt{zeros.h} (listado de código \ref{lst:zeros_h}) y \texttt{zeros.c}
    (listado de código \ref{lst:zeros_c}); y un algoritmo de comprobación 
    para saber si un número `$p$' de 64 bits es primo, implementado en el
    fichero \texttt{primes.h} (listado de código \ref{lst:primes_h}) y
    \texttt{primes.c} (listado de código \ref{lst:primes_c}).

    \item[\texttt{sync} --] un paquete que recoge funciones de sincronización
    entre distintos hilos de ejecución. Pese a que el dsPIC33E no cuenta
    con dicha funcionalidad, las rutinas de tratamiento de interrupciones
    se ejecutan en su propio contexto y podrían llegar a provocar una colisión
    con respecto al valor de ciertas variables.

    Este paquete se utiliza principalmente para conocer cuándo los tres motores
    que afectan a la posición del robot han finalizado su movimiento,
    implementando un algoritmo de exclusión mútua, desarrollado en los ficheros
    \texttt{mutex.h} (listado de código \ref{lst:mutex_h}) y \texttt{mutex.c}
    (listado de código \ref{lst:mutex_c}), y el algoritmo de barrera,
    escrito en los ficheros \texttt{barrier.h} (listado de código
    \ref{lst:barrier_h}) y \texttt{barrier.c} (listado de código \ref{lst:barrier_c}).

    \item[\texttt{timers} --] implementación de los \textit{timers} que
    gestionan la posición y movimiento de los tres motores. Ofrecen una
    interfaz común para poder gestionar la sincronización entre ellos así
    como cuándo se empieza un movimiento y cuándo se finaliza.

    El paquete se encuentra dividido en: \texttt{tmr3.h} (listado de código
    \ref{lst:tmr3_h}) y \texttt{tmr3.c} (listado de código \ref{lst:tmr3_c});
    \texttt{tmr4.h} (listado de código \ref{lst:tmr4_h}) y \texttt{tmr4.c}
    (listado de código \ref{lst:tmr4_c}); y \texttt{tmr5.h} (listado de código
    \ref{lst:tmr5_h}) y \texttt{tmr5.c} (listado de código \ref{lst:tmr5_c}).

    \item[\texttt{utils} --] distintas utilidades que se utilizan a lo largo
    de toda la aplicación. Se encuentran definidas utilidades para el manejo
    de \textit{buffers} de tamaño arbitrario, implementado en los ficheros
    \texttt{buffer.h} (listado de código \ref{lst:buffer_h}) y \texttt{buffer.c}
    (listado de código \ref{lst:buffer_c}); constantes matemáticas o 
    utilidades que se usan en tiempo de compilado por el preprocesador, 
    implementado en el fichero \texttt{defs.h}
    (listado de código \ref{lst:defs_h}); librería para el manejo del tiempo
    que ha pasado desde la ejecución de la aplicación, tanto en $ms$ como en
    $\mu s$, implementado en los ficheros \texttt{time.h} (listado de código
    \ref{lst:time_h}) y \texttt{time.c} (listado de código \ref{lst:time_c});
    definiciones de nuevos tipos de datos basados en estructuras, redefiniciones
    de tipos ya existentes y de constantes relacionadas a ellos, implementado en
    el fichero \texttt{types.h} (listado de código \ref{lst:types_h}); gestión
    de la salida estándar al puerto del microcontrolador para ser enviado vía
    \ac{UART}, implementado en los ficheros \texttt{uart.h} (listado de código
    \ref{lst:uart_h}) y \texttt{uart.c} (listado de código \ref{lst:uart_c});
    y finalmente distintas utilidades varias que no necesitan de ningún fichero
    específico para ellas ya que se consideran de carácter general, como puede
    ser la implementación de la rutina \texttt{foreach} en C, funciones de
    \textit{delay} según el tiempo especificado, comprobaciones con respecto
    a valores tipo \texttt{double} (por ejemplo, si son \texttt{NaN}) y utilidades
    para mapear un valor entre un límite superior e inferior, implementado en los
    ficheros \texttt{utils.h} (listado de código \ref{lst:utils_h}) y \texttt{utils.c}
    (listado de código \ref{lst:utils_c}).

    \item[\texttt{arm\_config.h} --] paquete que define las constantes físicas
    del brazo robótico, implementado en el fichero \ref{lst:arm_config_h}. Entre otros
    valores, se encuentran las longitudes de los segmentos inferior y superior del
    brazo (definidos anteriormente como $\overline{A_L}$ y $\overline{A_U}$),
    la altura de la base ($B_h$), etc.

    \item[\texttt{init} --] rutinas de configuración e inicialización de toda la
    placa, además de algunas funciones complementarias para permitir habilitar
    y deshabilitar interrupciones.

    En los ficheros \texttt{init.h} (listado de código \ref{lst:init_h}) e
    \texttt{init.c} (listado de código \ref{lst:init_c}) se encuentran rutinas para
    establecer la frecuencia de oscilación del reloj interno, los baudios a los
    que trabaja la \ac{UART}, los distintos \textit{timers} que se utilizan así como
    la inicialización de los pines de la placa y las interrupciones.

    \item[\texttt{interrupts} --] paquete que engloba múltiples rutinas de tratamiento
    de interrupciones del sistema.

    Implementadas en los ficheros \texttt{interrupts.h} 
    (listado de código \ref{lst:interrupts_h}) e \texttt{interrupts.c} (listado de
    código \ref{lst:interrupts_c}), destacan principalmente las rutinas de tratamiento
    de los fines de carrera así como las de gestión de la \ac{UART}, encargadas de
    asegurar una comunicación fiable entre ambos sistemas \ac{S1} y \ac{S2}.

    \item[\texttt{pragmas.h} --] implementación de la configuración básica del 
    microcontrolador. En dicho fichero (listado de código \ref{lst:pragmas_h}) se
    definen opciones como permitir la reasignación de puertos en tiempo de ejecución,
    \textit{overclock} de la velocidad del reloj, etc.

    \item[\texttt{s\_types.h} --] definición de tipos básicos del sistema
    para facilitar el manejo de ciertos registros, en particular, el registro
    \texttt{CORCON}. Implementado en el fichero \texttt{system\_types.h}
    (listado de código \ref{lst:system_types_h}).

    \item[\texttt{main.c} --] punto de entrada de la ejecución del \ac{SW} de
    \ac{S2}. Aúna todos los ficheros y paquetes mencionados anteriormente y los
    coordina para que el sistema se ejecute según se espera.

    Se compone principalmente de una rutina de inicialización (\texttt{setup()})
    y del bucle principal (\texttt{loop()}), donde configura el sistema para que
    respete los parámetros establecidos y gestiona los distintos eventos recibidos
    por la \ac{UART}. Configura además un modo \textit{cli} que permite la interacción
    directa con el sistema sin necesidad de una interfaz de usuario así como de un
    modo \textit{debug} que han de ser activados ambos en tiempo de compilación.

    Por otra parte, la gestión de los errores recae sobre este fichero así como
    el \textit{heartbeat} que mantiene activa la comunicación entre sistemas. Al ser
    además el coordinador de todo el sistema, las órdenes recibidas por la \ac{UART}
    son derivadas a otros paquetes para su interpretación y luego delegadas para
    realizar las órdenes especificadas, o indicar error en caso de que no sea un
    dispositivo verificado o si es una instrucción no implementada por el sistema.

    Se encuentra implementado en el fichero \texttt{main.c} (listado de código
    \ref{lst:main_c}).
\end{itemize}