Las comunicaciones entre los sistemas \ac{S2} y \ac{S1} se realiza mediante \ac{UART}
pero los mensajes no se componen de un lenguaje propio sino que se utiliza una
adaptación del GCode.

El GCode es el nombre que recibe habitualmente el lenguaje de programación más usado
de control numérico, el cual posee múltiples implementaciones \cite{Gcode2020}. La
estructura típica de un código en GCode se compone de una orden y una serie de
parámetros que pueden ser opcionales. Según la web de RepRap, se definen los siguientes
tipos de GCode \cite{GcodeEsRepRap}:

\begin{itemize}
    \item \texttt{Gnnn} -- Comando GCode estándar, como moverse hasta un punto.
    \item \texttt{Mnnn} -- Comando definido por RepRap, como encender un ventilador.
    \item \texttt{Tnnn} -- Seleccionar la herramienta nnn. En RepRap, las herramientas son extrusores.
    \item \texttt{Snnn} -- Parámetro de comando, como la tensión enviada a un motor.
    \item \texttt{Pnnn} -- Parámetro de comando, como el tiempo en milisegundos.
    \item \texttt{Xnnn} -- Una coordenada X, normalmente para moverse a ella. Puede ser un número entero o racional.
    \item \texttt{Ynnn} -- Una coordenada Y, normalmente para moverse a ella. Puede ser un número entero o racional.
    \item \texttt{Znnn} -- Una coordenada Z, normalmente para moverse a ella. Puede ser un número entero o racional.
    \item \texttt{Innn} -- Parámetro - Actualmente no utilizado.
    \item \texttt{Jnnn} -- Parámetro - Actualmente no utilizado.
    \item \texttt{Fnnn} -- \textit{Feedrate} en mm por minuto. (Velocidad de movimiento del cabezal de impresión).
    \item \texttt{Rnnn} -- Parámetro - usado para temperaturas.
    \item \texttt{Qnnn} -- Parámetro - Actualmente no utilizado.
    \item \texttt{Ennn} -- Longitud a extruir en mm. Ex exactamente como X, Y y Z, pero para la cantidad de filamento a extruir. Es común que los nuevos sistemas basados en pasos lo interpreten... Mejor: Skeinforge 40 y siguientes interpretan esto como la longitud absuluta de filamento insertado, no como la longitud de la extrusión que sale.
    \item \texttt{Nnnn} -- Número de línea. Utilizado para pedir la repetición de la transmisión en caso de errores de comunicación.
    \item \texttt{*nnn} -- Checksum. Usado para comprobar errores de comunicación.
\end{itemize}

En este proyecto se han utilizado los siguientes GCode:

\begin{itemize}
    \item \texttt{Gnnn} -- para indicar movimientos hasta un punto o coordenada angular.
    \item \texttt{Mnnn} -- para indicar órdenes, como detener el movimiento u obtener la posición actual.
    \item \texttt{Innn} -- órdenes relacionadas con el intercambio de mensajes en RSA.
    \item \texttt{Jnnn} -- órdenes que indican distintos tipos de errores durante la ejecución.
\end{itemize}

Durante la ejecución, los caracteres recibidos por la \ac{UART} son guardados en un
\textit{buffer} temporal de hasta 1024 caracteres. Una vez se recibe un salto de línea,
el \textit{buffer} es copiado a una cadena de caracteres de tamaño fijo, para no
emplear más memoria de la necesaria. Dicha acción activa además un \textit{flag}
que indica que un nuevo mensaje ha sido recibido, delegando su tratamiento al
bucle principal.

La gestión de los bytes recibidos se realiza en la rutina de tratamiento de
interrupción de la \ac{UART}:

\lstinputlisting[linerange={28-77}, firstnumber=28, caption=, style=C]{pArm-S2/pArm.X/interrupts.c}

y el GCode es interpretado en el paquete \texttt{gcode}, definido por los listados
de código \ref{lst:gcode_h} y \ref{lst:gcode_c}.

Actualmente, \ac{S2} interpreta y entiende las siguientes órdenes (tabla \ref{tab:gcode_def}):

\LTXtable{\linewidth}{software/s2/gcode_orders}

Cuando se recibe una orden esta es guardada junto con toda la información relativa a
ella en una estructura del tipo \texttt{order\_t}:

\lstinputlisting[linerange={110-138}, firstnumber=110, caption=, style=C]{pArm-S2/pArm.X/utils/types.h}

Dicha estructura contiene un \texttt{buffer\_t} con la orden recibida y un \textit{flag}
indicando si hay una nueva orden lista para ser interpretada. El bucle principal de 
ejecución comprueba en cada iteración si hay algún mensaje pendiente y, en caso afirmativo,
delega la interpretación al paquete \texttt{gcode} (listados de código \ref{lst:gcode_h}
y \ref{lst:gcode_c}). Una vez interpretada la línea recibida, se devuelve el control
al bucle principal y se realizan las acciones oportunas.
