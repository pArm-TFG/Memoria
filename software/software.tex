El proyecto tiene una clara división del \ac{SW} en función del dispositivo donde este será ejecutado. Por un lado, se ha desarrollado un \ac{SW} de alto nivel en Python, el cual permite la interacción directa de un operador humano con el brazo robótico mediante una interfaz de usuario. Este \ac{SW} gobierna \ac{S1}. 

Por otro lado, y de manera concurrente, se ha desarrollado un \ac{SW} que será cargado en la placa de control, cuyo propósito es generar las señales necesarias para mover los motores, e interpretar las órdenes de movimiento que lleguen de \ac{S1}. Este \ac{SW} gobierna \ac{S2}.

Para comunicar estos dos sistemas se ha desarrollado un pseudo lenguaje basado en GCode el cual servirá para agilizar las comunicaciones y simplificar el envío y recepción de datos como posiciones o mensajes de error.

Cabe destacar que la comunicación es completamente asíncrona. Esto se debe a que los dos sistemas se comunican mediante el estándar UART, el cual es asíncrono, y permite además poder seguir comunicando los dos sistemas mientras estos realizan otras tareas distintas a la comunicación. Por ejemplo, \ac{S2} debe ser capaz de poder escuchar el puerto UART mientras mueve los motores. Por otro lado, \ac{S1} debe ser capaz de seguir permitiendo la interacción con la interfaz de usuario mientras está mandando una orden a \ac{S1}.
