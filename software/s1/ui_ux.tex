Para explicar el software que correrá en el \ac{S1} se va a hacer una división entre la parte que implementa la lógica de control y comunicaciones y la parte que implementa la interfaz gráfica.

\subsection{Lógica}

La lógica de control se ha dividido en varios paquetes para encapsular distintas funcionalidades del sistema por separado, estos paquetes son: ''comunications'', ''control'', ''gcode'', ''logger'', ''security'' y ''utils''.

A continuación se procede a explicar la funcionalidad de cada uno de los paquetes a un nivel de abstracción alto. Posteriormente, se explicarán los archivos .py que cada uno de los paquetes contiene.

Además de las explicaciones dadas en esta sección, existen comentarios en el código que explicarán mas en detalle cada una de las funciones.

A continuación se procede a explicar cada uno de los paquetes.

\subsubsection{Comunications}
Este paquete se encarga de gestionar la comunicaciones desde y hacia \ac{S1}. Sirve principalmente para facilitar la operación y la lectura del buffer de recepción.
También permite la escritura por un puerto UART.

Contiene un solo archivo, ''connection.py'', en el cual están definidas todas las funciones relacionadas con la lectura y la escritura a través del puerto UART.

Cabe destacar que para poder utilizar este puerto para la comunicación se ha hecho uso de la biblioteca ''serial'' de python.

\subsubsection{Control}

Este paquete contiene la lógica de control propiamente dicha. Sirve para implementar los métodos principales de movimiento, de comunicación y de  ''handshake''. Además ofrece una interfaz lógica para que la interfaz de usuario pueda, a través de los botones que aparecen en pantalla, comunicarse con la lógica de control.

Contiene 3 archivos.py:

\begin{itemize}
    \item control.py: El cual implementa los métodos de movimiento y petición de posiciones. Es el archivo principal de control y sus funciones desencadenan llamadas a todos los demás paquetes para poder realizar funciones complejas.
    \item control\_interface.py: Es una interfaz que implementa parte de los métodos de control.py. Este archivo es utilizado por la GUI para poder desencadenar acciones en \ac{S1} a partir de la interacción del operario con la interfaz gráfica.
    \item control\_management.py: Ofrece funciones auxiliares que permiten a control.py monitorizar que las ordenes enviadas a \ac{S2} son realizadas con éxito o en caso contrario, controlar los errores que se pudieran producir.
\end{itemize}

Cabe destacar que en este paquete se han utilizado librerías  que implementan el uso de futuros en Python con el objetivo de poder monitorizar el movimiento de \ac{S2} sin necesidad de bloquear la interfaz de usuario.

\subsubsection{Gcode}

Aquí se encuentran la lógica de interpretación y generación de las tramas de GCODE que se transmiten entre \ac{S1} y \ac{S2}.

Contiene 2 archivos .py:

\begin{itemize}
    \item generator.py: Este archivo contiene las funciones que, a partir de los valores que reciben como parámetros, generan las tramas en el formato adecuado para ser enviadas.
    
    \item interpreter.py: Este archivo sirve para analizar gramaticalmente los bytes que hay en el buffer de \ac{S1}.
    Para poder realizar esta labor, transforma los bytes en cadena de caracteres y posteriormente analiza dichas cadenas y las interpreta.
\end{itemize}

Se ha usado la librería typing para permitir el uso de distintos tipos de datos que simplifiquen el tratamiento de los datos y la interpretación de las tramas.

\subsubsection{Logger}

Este paquete permite llevar un registro de los datos y las acciones importantes que ocurren en \ac{S1}. Genera un archivo en el que se guardan diferentes datos para poder hacer debugging y crashlogs.

Contiene un único archivo, logger.py, el cual alberga las funciones necesarias para crear el archivo y operar con el, además de dar un formato estándar a las diferentes trazas.

En este paquete destaca el uso de la librería ''logging'' para poder generar nombres únicos para los distintos archivos de registro y además para comprimir de manera automática los archivos de registro antiguos.

\subsubsection{Security}

A través de este paquete el \ac{S1} consigue autenticar al sistema \ac{S2} y viceversa.

Contiene un solo archivo, rsa.py, el cual alberga las funciones necesarias para que, a partir de los números procedentes de \ac{S2}, se pueda ''desfirmar'' la trama recibida y poder reenviarla en claro posteriormente.

Se ha usado la librería typing para poder usar tipos de datos que simplificaran el proceso.


\subsubsection{Utils}

En este paquete se encuentran todos los archivos auxiliares que no se puedan ubicar en ningún otro paquete.

Solo contiene un archivo, error\_data.py, el cual simplemente implementa un tipo de dato creado especialmente para poder contener de una manera mas organizada los datos referentes a los errores provenientes de \ac{S2}.

Se emplea la librería collections para permitir el uso de tipos de datos auxiliares.


