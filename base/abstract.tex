\begin{abstract}
    En este documento se va a tratar el desarrollo del \pArm{}, un proyecto
    integral de ingeniería en el que se modela, diseña y construye un brazo
    robótico utilizando tecnología de impresión 3D como base. El objetivo
    principal es ofrecer una forma asequible y sencilla para que otra
    persona pueda replicar el proyecto y adentrarse en el mundo de la robótica
    por su cuenta.

    Para ello, primero se elicitarán los requisitos que permitirán posteriormente
    modelar y diseñar el sistema de forma fiel. A su vez, se estudiarán las
    características del sistema \textit{hardware} lo que permitirá desarrollar
    y construir una placa de control que será la encargada de gestionar los
    movimientos del brazo robótico.

    Además, las fases de diseño anteriores simplifican el proceso de desarrollo
    del \textit{software} que ejecutarán los sistemas y permitirán abordar el modelo
    matemático que rige la estructura pantográfica del brazo robótico atendiendo a
    las limitaciones tanto físicas como del sistema propuesto en sí.

    Por otro lado, se modelarán y diseñarán nuevas piezas que permitirán construir el
    brazo robótico con otros tipos de componentes distintos a los del brazo original
    así como con la nueva placa de control.

    Por último, se proponen futuras líneas de mejora que se consideran interesantes
    a la hora de completar el proyecto. Se incluyen además en los anexos el código
    fuente de las aplicaciones desarrolladas ya que se referencia directamente a lo
    largo del presente documento.
\end{abstract}

\selectlanguage{english}
\begin{abstract}
    The \pArm{} development, an integral engineering project which models, designs and
    builds a robotic arm using 3D printing technology as a basis is going. The main objective is to offer an affordable
    and easy way for everyone to replicate this project so they can introduce
    themselves within robotics community.

    Firstly, the requirements will be elicited which, will allow the modeling and the 
    design of the system in further steps of the development. Concurrently, the
    hardware characteristics will be studied in order to allow the development and
    construction of the board that will handle the movements of the robotic arm.

    In addition, the mentioned design steps simplify the software development process
    alongside the mathematical model, which is defined by both the physical structure
    itself and the proposed system.

    On the other hand, new pieces will be designed in order to make the robotic
    manipulator compatible with new external componentes, compared to the 
    original ones used in the $\mu$Arm, and the new designed board.

    Finally, new improvements that the team considers interesting to complete
    will be proposed. The annexes are included with the source code developed for 
    the applications, as they are directly referenced in the present document.
\end{abstract}

\selectlanguage{spanish}