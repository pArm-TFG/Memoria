\begin{acronym}
    \acro{SDK}{\textit{Software Development Kit}}
    \acro{ROS}{\textit{Robot Operating System}}
    \acro{API}{\textit{Application Programming Interface}}
    \acro{SW}{\textit{software}}
    \acro{HW}{\textit{hardware}}
    \acro{pArm}{\textit{Printed -- Arm}}
    \acro{USB}{\textit{Universal Serial Bus}}
    \acro{ODS}{Objetivos de Desarrollo Sostenible}
    \acro{OS}{\textit{Open--Source}}
    \acro{OH}{\textit{Open--Hardware}}
    \acro{S1}{Sistema 1 -- ordenador}
    \acro{S2}{Sistema 2 -- \pArm{}}
    \acro{GUI}{\textit{Graphical User Interface}}
    \acro{GTK}{\textit{GIMP Toolkit}}
    \acro{SoC}{\textit{System On Chip}}
    \acro{PWM}{\textit{Pulse--Width Modulation}}
    \acro{GPIO}{\textit{General Purpose Input/Output}}
    \acro{UART}{\textit{Universal Asynchronous Receiver--Transmitter}}
    \acro{RAM}{\textit{Random Access Memory}}
    \acro{ADC}{\textit{Analog--Digital Conversor}}
    \acro{PLA}{Ácido Poliláctico}
    \acro{ABS}{Acrilonitrilo Butadieno Estireno}
    \acro{DSP}{\textit{Digital Signal Processor}}
    \acro{PLL}{\textit{Phase Loop Lock}}
    \acro{THT}{\textit{Through-Hole Technology}}
    \acro{SMD}{\textit{Surface--Mount Device}}
    \acro{PCB}{\textit{Printed Circuit Board}}
    \acro{ALU}{\textit{Arithmetic--Logic--Unit}}
    \acro{CPE}{Copoliéster}
    \acro{PVA}{Acetato de polivinilo}
\end{acronym}

\begin{itemize}
    \item \ac{SDK} -- colección de herramientas de desarrollo \ac{SW}.
    \item \textit{hand--shake} -- en informática, negociación entre pares para establecer de forma dinámica los parámetros de un canal de comunicaciones. 
    \item \ac{ROS} -- conjunto de librerías \ac{SW} que ayudan a construir aplicaciones para robots.
    \item \textit{Firmware} -- \ac{SW} programado que especifica el orden de ejecución del sistema.
    \item \ac{GUI} -- siglas que significan ``Interfaz Gráfica de Usuario'' (en castellano).
    \item \ac{GTK} -- biblioteca de componentes gráficos multiplataforma para desarrollar interfaces gráficas de usuario.
    \item \ac{SoC} -- tecnología de fabricación que integra todos o gran parte de los módulos, de un sistema en un circuito integrado. 
    \item \ac{PWM} -- Señal cuadrada de periodo habitualmente constante, entre flancos de subida, en la que se modula el tiempo a nivel alto
    \item \ac{GPIO} -- pin genérico cuyo comportamiento puede ser controlado en tiempo de ejecución.
    \item \ac{UART} -- estándar de comunicación dúplex.
    \item Dúplex -- término que define a un sistema que es capaz de mantener una comunicación bidireccional, enviando y recibiendo mensajes de forma simultánea.
    \item Widget -- la parte de una GUI (interfaz gráfica de usuario) que permite al usuario interconectar con la aplicación.
    \item \ac{RAM} -- memoria volátil que permite operaciones de acceso aleatorio.
    \item \textit{Deep--Sleep} -- estado de un microcontrolador en el cual consume muy poca cantidad de energía.
    \item \textit{bit} -- unidad mínima de información de un computador digital.
    \item \ac{THT} -- tecnología que utiliza agujeros pasantes que se practican en las placas de los circuitos impresos para el montaje de diferentes elementos electrónicos.
    \item \ac{SMD} -- tecnología que utiliza componentes de montaje superficial para la inserción de diferentes elementos electrónicos en un circuito impreso.
    % \item \ac{PCB} -- Placa de circuito impreso.
\end{itemize}
